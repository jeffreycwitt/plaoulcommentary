
        %this tex file was auto produced from TEI by lombardpress-print on 2016-01-10T14:54:00.792-05:00 using the  file:/Users/JCWitt/Desktop/lombardpress-print/lbp-latex-critical.xsl 
        \documentclass[twoside, openright]{report}
        
        % etex package is added to fix bug with eledmac package and mac-tex 2015
        % See http://tex.stackexchange.com/questions/250615/error-when-compiling-with-tex-live-2015-eledmac-package
        \usepackage{etex}
        
        %imakeidx must be loaded beore eledmac
        \usepackage{imakeidx}
        
        \usepackage{eledmac}
        \usepackage{titlesec}
        \usepackage [english]{babel}
        \usepackage [autostyle, english = american]{csquotes}
        \usepackage{geometry}
        \usepackage{fancyhdr}
        \usepackage[letter, center, cam]{crop}
        
        
        \geometry{paperheight=10in, paperwidth=7in, hmarginratio=3:2, inner=1.7in, outer=1.13in, bmargin=1in} 
        
        %fancyheading settings
        \pagestyle{fancy}
        
        %git package 
        \usepackage{gitinfo2}
        
        %watermark
        \usepackage{draftwatermark}
        
        
        %quotes settings
        \MakeOuterQuote{"}
        
        %title settings
        \titleformat{\section} {\normalfont\scshape}{\thesection}{1em}{}
        \titlespacing\section{0pt}{12pt plus 4pt minus 2pt}{12pt plus 2pt minus 2pt}
        \titleformat{\chapter} {\normalfont\Large\uppercase}{\thechapter}{50pt}{}
        
        %eledmac settings
        \foottwocol{B}
        \linenummargin{outer}
        \sidenotemargin{inner}
        
        %other settings
        \linespread{1.1}
        
        %custom macros
        \newcommand{\name}[1]{\textsc{#1}}
        \newcommand{\worktitle}[1]{\textit{#1}}
        
        
        \begin{document}
        \fancyhead[RO]{Lectio 20, de Notitia}
        \fancyhead[LO]{Petrus Plaoul}
        \fancyhead[LE]{0.1.0+dev+\gitDescribe}
        \chapter*{Lectio 20, de Notitia}
        \addcontentsline{toc}{chapter}{Lectio 20, de Notitia}
        
         
        \beginnumbering
         \section*{Lectio 20, de Notitia} 
        \bigskip
         \section*{[Recapitulatio prologi]} 
        \pstart
        \ledsidenote{\textbf{1}}
        Sciendum est, secundum quod in praecedenti lectione dicebatur, in toto processu facto circa prologum visum est qualiter fides non subiacet humano iudicio. Et consequenter deductum est quod investigatio humana subiacet fidei et per eam dirigitur. Dictum est \edtext{ulterius}{\Bfootnote{\textit{om.} V n1}} quod theologia participat \edtext{fidei}{\Bfootnote{fide V n2}} et investigationis humanae \edtext{rationem,}{\Bfootnote{\textit{om.} V n3}} ideo non participat fidei \edtext{rationem}{\Bfootnote{\textit{om.} V n4}} ad concedendum proprie, \edtext{sed}{\Bfootnote{subdeligabilis et dabitur sibi audientia \textit{add. sed del.} SV n5}} ad dirigendum et curandum humanam \edtext{investigationem}{\Bfootnote{fidei \textit{in textu} V n6}} ab erroribus et falsis apparentiis contra veritatem, ideo fides \edtext{sibi}{\Bfootnote{arguit \textit{add. sed del.} SV n7}} subdelegabit et dabitur sibi audientia ut \edtext{posset}{\Bfootnote{possit S V n8}} inquirere veritatem.
        \pend
      
        \bigskip
         \section*{[Rationes Principales: quod Deus non est a nobis cognoscibilis]} 
        \bigskip
         
        \pstart
        \ledsidenote{\textbf{2}}
        Et primo circa primum principium \edtext{positum,}{\Bfootnote{\textit{om.} V n9}} scilicet \enquote*{Deum esse,} arguit investigatio humana et adducit testes \edtext{secum}{\Bfootnote{sicut V n10}} contra fidem primo negantium \enquote*{Deum \edtext{esse}{\Bfootnote{\textit{om.} V n11}}} \edtext{nam \edtext{dicit}{\Bfootnote{dixti S n13}} \edtext{insipiens}{\Bfootnote{incipiens S n14}} in corde suo \enquote*{non est Deus}}{\lemma{nam \dots\ Deus}\Bfootnote{\textit{om.} V n12}} \edtext{et}{\Bfootnote{\textit{om.} S  etiam V n15}} \edtext{alios \edtext{adducit}{\Bfootnote{adduxit S n17}} philosophos dubitantes \enquote*{Deum esse.}}{\Bfootnote{\textit{om.} V n16}} Alios ponentes ipsum esse mere corporalem, scilicet, caelum \edtext{et}{\Bfootnote{\textit{om.} V n18}} astra et huiusmodi.
        \pend
     
        \pstart
        \ledsidenote{\textbf{3}}
        Consequenter ex dictis fidei, arguit quod Deus non est a nobis cognoscibilis, \ledsidenote{V39ra} et quod \edtext{ita}{\Bfootnote{\textit{om.} S n19}} sic testatur Scriptura Sacra, \edtext{quia scriptum est}{\Bfootnote{\textit{om.} R SV S n20}} \edtext{\enquote{Deum nemo vidit umquam.}}{\Afootnote{Ioannes 1:18}} Etiam \name{Apostolus}\index[persons]{Paul, the Apostle} dicit quod \edtext{\enquote{\edtext{Deus}{\Bfootnote{\textit{om.} R SV S n21}} inhabitat lucem inaccessibilem.}}{\Afootnote{I Timotheus 6:16}} \name{Dionysius}\index[persons]{Pseudo-Dionysius} \ledsidenote{R29vb} etiam dicit \edtext{de ipso}{\Bfootnote{\textit{om.} V n22}} quod \edtext{\enquote{de ipso non est scientia nec opinio nec phantasia, ergo non est a nobis cognoscibilis.}}{\lemma{de \dots\ cognoscibilis.}\Afootnote{Ps-Dionysius, XXX}} 
        \pend
      
        \bigskip
         \section*{[Prima Ratio]} 
        \pstart
        \ledsidenote{\textbf{4}}
        Deinde arguitur per rationes probando quod Deus non est a nobis cognoscibilis, quia, si sic, sit ergo cognitio \edtext{ipsius}{\Bfootnote{\textit{om.} S V n23}} Dei a; \edtext{vel}{\Bfootnote{\textit{om.} V n24}} \edtext{igitur}{\Bfootnote{tunc V n25}} ipsa est proportionata obiecto et \edtext{tunc}{\Bfootnote{\textit{om.} V n26}} est \edtext{infinita,}{\Bfootnote{infinitum R SV n27}} \edtext{cum obiectum \edtext{est}{\Bfootnote{sit S n29}} infinitum,}{\Bfootnote{\textit{om.} V n28}} et sic anima humana non est ipsius capax; vel est proportionata potentiae, et sic solum est finita et improportionata obiecto. Et per consequens non erit notitia Dei.
        \pend
      
        \bigskip
         \section*{[Secunda ratio]} 
        \pstart
        \ledsidenote{\textbf{5}}
        Secundo \ledsidenote{SV213vb} potest argui per comparationem loci ad \edtext{locabile.}{\Bfootnote{locabilem SV  locale V n30}} Nam si esset aliquod mobile localiter distans infinite a suo loco in nullo tempore finito, immo numquam posset moveri in illum locum nisi per formam infinitae virtutis. Modo loquendo proportionaliter de loco spirituali, Deus est locus spiritualis animarum infinite distans ab anima propter infinitam distantiam \edtext{obiectorum}{\Bfootnote{eorum V n31}} ergo anima non \edtext{potest eum}{\Bfootnote{poterit ipsum S n32}} attingere cognitive, nisi per notitiam infinitam simpliciter, quae est nobis impossibilis \edtext{simpliciter.}{\Bfootnote{hic in via V n33}} 
        \pend
      
        \bigskip
         \section*{[Quarta ratio]} 
        \pstart
        \ledsidenote{\textbf{6}}
        Deinde probatur \edtext{quod}{\Bfootnote{\textit{om.} R SV n34}} etiam nec adiutorio \edtext{luminis gloriae \edtext{posset}{\Bfootnote{possit V n36}} anima}{\Bfootnote{fidei potest anima sive luminis gloriae S n35}} ad huiusmodi notitiam \edtext{seu}{\Bfootnote{vel V n37}} \ledsidenote{S35va} cognitionem ipsius Dei elevari, quia vel huiusmodi lumen esset divina essentia vel \edtext{aliquod}{\Bfootnote{aliquid S V n38}} creatum sufficienter elevans \edtext{potentiam}{\Bfootnote{potentiae V n39}} animae cognitivam ad huiusmodi cognitionem Dei; non potest dici quod sit divina essentia, quia tunc infinite immutaret animam.
        \pend
     
        \pstart
        \ledsidenote{\textbf{7}}
        Pro cuius \edtext{determinatione,}{\Bfootnote{declaratione S V n40}} supponitur quod quanto aliqua notitia est perfectioris speciei tanto perfectius immutat. \edtext{Patet illud clare \edtext{quare}{\Bfootnote{quasi V n42}} inductive; fundatur etiam \edtext{in}{\Bfootnote{\textit{om.} V n43}} ratione, quia perfectio notitiae sumitur penes hoc, quod perfectori modo immutat,}{\lemma{Patet \dots\ immutat,}\Bfootnote{\textit{om.} S n41}} ita quod a \edtext{priori}{\Bfootnote{posteriori V n44}} perfectio notitiae attenditur penes hoc, quod est perfectioris speciei et a posteriori perfectio speciei attenditur penes \edtext{hoc quod est}{\Bfootnote{\textit{om.} R SV V n45}} perfectius immutare.
        \pend
     
        \pstart
        \ledsidenote{\textbf{8}}
        Modo considerata tota latitudine cognitionum creaturarum, quae in infinitum procedit \edtext{ascendendo}{\Bfootnote{
                        semper \textit{in textu} V n46}} secundum perfectionem specificam, notitia ipsius Dei est supra \edtext{totam}{\Bfootnote{creatam \textit{corr. ex} creaturam V n47}} latitudinem notitiarum creatarum, \edtext{et est perfectioris speciei quam tota latitudo notitiarum creaturarum,}{\Bfootnote{\textit{om.} V n48}} et \edtext{tamen}{\Bfootnote{non \textit{in textu} V n49}} tota illa latitudo infinite immutat; ergo notitia Dei perfectiori modo quam infinite immutaret.
        \pend
     
        \pstart
        \ledsidenote{\textbf{9}}
        Si \edtext{daretur}{\Bfootnote{et \textit{in textu} V n50}} ista est \ledsidenote{V39rb} fortior \edtext{ratio}{\Bfootnote{in materia \textit{in textu} V n51}} tenentium quod Deus \edtext{non}{\Bfootnote{\textit{om.} V n52}} potest esse notitia creaturae et supplere vicem speciei et volitionis. Et \edtext{\name{Magister Ioannes de Ripa}\index[persons]{}}{\Afootnote{Ripa, XXX}} dicit quod Deus supplens vicem speciei et \edtext{volitionis}{\Bfootnote{et \textit{in textu} V n53}} est notitia infinita quoad speciem, et \edtext{finita}{\Bfootnote{infinita R n54}} quoad gradum.
        \pend
     
        \pstart
        \ledsidenote{\textbf{10}}
        Sed mihi videtur cum \edtext{sui}{\Bfootnote{sua S n55}} reverentia quod istud non sit \edtext{multum}{\Bfootnote{\textit{om.} V n56}} bene dictum \edtext{in materia,}{\Bfootnote{\textit{om.} V n57}} quia eo ipso quod est infinita \ledsidenote{R30ra} \edtext{quoad}{\Bfootnote{secundum V n58}} speciem, est infinita simpliciter, et sic infinita quoad gradum. Sed non \edtext{oporteret}{\Bfootnote{oportet V n59}} e converso, si esset infinita quoad gradum, \edtext{quod esset infinita simpliciter. Si enim esset infinita albedo quoad gradum, non propter hoc esset infinita simpliciter.}{\lemma{quod \dots\ simpliciter.}\Bfootnote{\textit{om.} V n60}} Verbi gratia: anima intellectiva est praecise finita et tamen in latitudine entium esset perfectior quam infinita albedo, si \edtext{daretur.}{\Bfootnote{et \textit{in textu} V n61}} Ideo solutio non videtur \edtext{vera,}{\Bfootnote{nam V n62}} quia sicut tactum est perfectio notitiae simpliciter magis attenditur quoad speciem quam quoad gradum. Modo \edtext{per}{\Bfootnote{ni \textit{add. sed del.} SV n63}} ipsum notitia divina supplens vicem speciei est infinita quoad speciem, ergo infinita simpliciter ergo infinite immutat nec potest dici quod lumen gloriae elevans animam sit aliquod creatum ex eo \edtext{quod}{\Bfootnote{\textit{om.} V n64}} divina essentia est immensa et incomprehensibilis a creatura, ergo non potest esse aliqua species vel notitia creata ipsam essentiam divinam secundum quodlibet \edtext{sui}{\Bfootnote{simpliciter V n65}} repraesentans.
        \pend
     
        \pstart
        \ledsidenote{\textbf{11}}
        Et in istis rationibus, \edtext{quandoque}{\Bfootnote{quandoque \textit{corr. ex} quandocumque R  quandocumque SV n66}} inducam de notitia beatifica, quandoque de notitia hic in via, et videbitur in \edtext{solutione}{\Bfootnote{solutionibus V n67}} istarum rationum de utraque notitia.
        \pend
      
        \bigskip
         \section*{[Quinta ratio]} 
        \pstart
        \ledsidenote{\textbf{12}}
        Item quinto \ledsidenote{SV214ra} notitia talis \edtext{esset}{\Bfootnote{est R n68}} simpliciter infinita. Istud patet proportionando notitias secundum proportionem obiectorum. Nam generaliter perfectioris obiecti est perfectior cognitio, et specialiter loquendo de \ledsidenote{S35vb} notitia intuitiva. \edtext{Modo}{\Bfootnote{nam S n69}} latitudo notitiarum creatarum est infinita, scilicet, continue ascendendo secundum notitiarum perfectionem, quia quacumque data potest dari in duplo perfectior in \edtext{triplo,}{\Bfootnote{triangulo SV n70}} et sic in infinitum, \edtext{sicut}{\Bfootnote{sic V n71}} contingit de obiectis ipsarum notitiarum. Modo, obiectum increatum, \edtext{scilicet}{\Bfootnote{si S n72}} ipse Deus, est \edtext{perfectior}{\Bfootnote{perfectius V n73}} quam tota \edtext{latitudo}{\Bfootnote{infinita \textit{in textu} V n74}} \edtext{obiectorum \edtext{creatorum,}{\Bfootnote{esset \textit{in textu} V n76}} ergo notitia ipsius \edtext{Dei}{\Bfootnote{\textit{om.} R SV n77}} erit perfectior quam notitia infinitorum obiectorum creatorum,   \edtext{et tamen notitia infinitorum}{\Bfootnote{\textit{om.} V n78}}}{\lemma{obiectorum \dots\ cross-nesting}\Bfootnote{\textit{om.} S n75}} \edtext{obiectorum creatorum,}{\Bfootnote{\textit{om.} V n79}}  si esset, \edtext{esset}{\Bfootnote{\textit{om.} V esset \textit{add.} \textit{interl.} SV n80}} infinita, ergo etc.
        \pend
     
        \pstart
        \ledsidenote{\textbf{13}}
        Et si dicatur, \edtext{quod}{\Bfootnote{
                          duae
                         V n81}} non est eadem proportio cognitionum inter se, sicut obiectorum ita quod non \edtext{semper}{\Bfootnote{\textit{om.} S n82}} oportet quod obiecti perfectioris \ledsidenote{V39va} in duplo sit \edtext{notitia}{\Bfootnote{cognitio S n83}} perfectior in duplo, retorquetur \edtext{ratio}{\Bfootnote{ideo S n84}} iterum. Verbi gratia, datis duabus notitiis, quarum una est dupla ad aliam, si \edtext{dicas}{\Bfootnote{dicis V n85}} quod \edtext{ad causandam}{\Bfootnote{eandem V n86}} secundam non sufficit obiectum duplae perfectionis, si enim non sufficiat, \edtext{capiam}{\Bfootnote{copiam SV n87}} obiectum in centuplo vel millecuplo perfectius quod sufficiet causare notitiam in duplo perfectiorem. Et sic \edtext{augendo}{\Bfootnote{augebo S V n88}} continue proportiones obiectorum intantum quod semper \edtext{duplabitur}{\Bfootnote{dupla videtur V n89}} notitia, cum sit processus in infinitum in \edtext{obiectis continue ascendendo secundum perfectionem specie. Et ita erit processus in infinitum in}{\lemma{obiectis \dots\ in}\Bfootnote{\textit{om.} V n90}} notitiis \ledsidenote{R30rb} semper \edtext{duplando}{\Bfootnote{duplicando V n91}} notitias et in multis habet locum ista augmentatio proportionum.
        \pend
      
        \bigskip
         \section*{[Sexta ratio]} 
        \pstart
        \ledsidenote{\textbf{14}}
        Item talis Dei notitia esset perfectior quam esset una alia \edtext{quae esset}{\Bfootnote{\textit{om.} V n92}} infinitorum obiectorum creatorum, et cuiuslibet distincte, \edtext{sed}{\Bfootnote{si V n93}} illa esset infinita, ergo etc. Et istam rationem facit \edtext{\name{Adam,}\index[persons]{Adam Wodeham}}{\Bfootnote{idem R SV n94}} \edtext{quod autem notitia ipsius Dei esset perfectior quam notitia infinitorum obiectorum creatorum, et cuiuslibet distincte, si essent.}{\lemma{quod \dots\ essent.}\Bfootnote{\textit{om.} V n95}} \edtext{Patet}{\Bfootnote{Antecedens patet V n96}} patet quia Deus est perfectius obiectum in se quam tota \edtext{multitudo}{\Bfootnote{latitudo V n97}} \edtext{infinitorum}{\Bfootnote{\textit{om.} V n98}} obiectorum creatorum, si esset.
        \pend
      
        \bigskip
         \section*{[Septima Ratio]} 
        \pstart
        \ledsidenote{\textbf{15}}
        Septimo \edtext{sic,}{\Bfootnote{\textit{om.} V n99}} quia \edtext{vel}{\Bfootnote{\textit{om.} R SV S n100}} per talem cognitionem intuitivam Deus \edtext{apparet}{\Bfootnote{appareret V n101}} bonum finitum vel infinitum. Non \edtext{potest dici quod apparet bonum finitum,}{\Bfootnote{primum V n102}} quia tunc aliqua creatura posset apparere intuitive melior Deo \edtext{et}{\Bfootnote{\textit{add. sed del.} S n103}} consequenter diligibilior, et consequenter sic videns posset \edtext{rationaliter}{\Bfootnote{rationabiliter V n104}} Deo praeferre creaturam. \edtext{Non}{\Bfootnote{Nec V n105}} potest dici quod \edtext{apparet}{\Bfootnote{appareret V n106}} bonum infinitum, quia \edtext{si sic}{\Bfootnote{\textit{om.} R SV S n107}} tunc comprehenderetur, quia quia cognosceretur secundum quodlibet sui.
        \pend
      
        \bigskip
         \section*{[Octava ratio]} 
        \pstart
        \ledsidenote{\textbf{16}}
         \edtext{Octavo}{\Bfootnote{Item R SV  Sexto S n108}} potest argui per comparationem distantiae localis obiecti sensibilis a sensu. Nam propter \edtext{magnam}{\Bfootnote{materiam V n109}} distantiam, sensus decipitur circa obiectum \edtext{proprium}{\Bfootnote{suum S  suum proprium V n110}} et iudicat rem minorem esse quam sit; ita proportionaliter de visu intellectuali propter infinitam distantiam decipitur circa obiectum suum infinite \ledsidenote{S36ra} ab eo distans. Et per consequens non possumus habere veram notitiam de \edtext{ipso}{\Bfootnote{\textit{om.} S n111}} Deo.
        \pend
     
        \pstart
        \ledsidenote{\textbf{17}}
        Item ex parte obiecti: si esset aliquod luminosum infinitum quod infinite distaret a visu, visus non iudicaret \enquote*{ipsum esse infinitum}, ergo ita erit de visu spirituali.
        \pend
     
        \pstart
        \ledsidenote{\textbf{18}}
        Item etiam ex parte \edtext{potentiae.}{\Bfootnote{\textit{om.} V n112}} Nam debilitas potentiae facit apparere obiectum remissius \edtext{vel}{\Bfootnote{\textit{om.} V n113}} et minus quam sit, ergo si potentia in infinitum distet ab obiecto in infinitum deficiet a \edtext{veri}{\Bfootnote{natura V n114}} obiecti cognitione. Illud \ledsidenote{SV214rb} patet de oculo male disposito, \ledsidenote{V39vb} \edtext{ut in \edtext{caecutiente,}{\Bfootnote{caecitate, V n116}}}{\Bfootnote{\textit{om.} R SV n115}} \edtext{cui}{\Bfootnote{cuius R \textit{om.} V n117}} lux solis videtur remississima. Istud etiam \edtext{patet}{\Bfootnote{apparet SV V n118}} \edtext{de}{\Bfootnote{in V n119}} oculo \edtext{nocti}{\Bfootnote{niti SV S V n120}} coracis. Unde propter debilitatem sui visus iudicat lumen solis remississimum intantum quod iudicat ipsum non sufficere ad videndum.
        \pend
     
        \pstart
        \ledsidenote{\textbf{19}}
        Item dato quod Deus esset \edtext{imaginabilis vel conceptibilis, non non sequitur quod Deus esset. Sicut enim non sequitur: vacuum est imaginabile vel chimera vel impossibile, ergo vacuum est, etc., quia sicut dicit  \name{Philosophus}\index[persons]{Aristotle} III \worktitle{Physicorum}\index[works]{Physics}  circa finem: \edtext{\enquote{Imaginatum non est credendum.}}{\Afootnote{Aristoteles, Physica, III}} \edtext{Et}{\Bfootnote{da \textit{add. sed del.} R n122}} per consequens, dato quod primae rationes \edtext{non}{\Bfootnote{nam R  vero SV n123}} concluderent quin Deum esset}{\lemma{imaginabilis \dots\ esset}\Bfootnote{\textit{om.} V n121}} imaginabilis vel conceptibilis, non propter hoc sequitur quod Deus esset. Omnis enim ratio \edtext{facta}{\Bfootnote{\textit{om.} V n124}} ad probandum \enquote*{quod Deus sit} est \edtext{ita}{\Bfootnote{in S n125}} apparenter solubilis, sicut ipsa apparenter arguit, ita quod considerata \edtext{tota}{\Bfootnote{\textit{om.} V n126}} latitudine \edtext{apparentiarum}{\Bfootnote{solutionum illarum \textit{add. sed del.} R n127}} \edtext{rationum}{\Bfootnote{ex alia \textit{add. sed del.} R n128}} \ledsidenote{R30va} \edtext{parte latitudo apparentiarum rationum in solutionibus probantium}{\Bfootnote{probabilis V n129}} \enquote*{Deum esse} ex una parte et considerata \edtext{tota}{\Bfootnote{\textit{om.} V n130}} latitudine apparentiarum solutionum illarum rationum ex alia parte, latitudo \edtext{apparentiarum in solutionibus esse aequalis latitudini}{\Bfootnote{\textit{om.} V n131}} \edtext{apparentiarum}{\Bfootnote{illarum \textit{in textu} V n132}} rationum \edtext{probantium}{\Bfootnote{principium V n133}} \enquote*{Deum esse}, et per consequens una pars non debet magis movere quam alia.
        \pend
      
        \bigskip
         \section*{[Nona ratio]} 
        \pstart
        \ledsidenote{\textbf{20}}
        Item \enquote*{Deum esse} \edtext{traditur}{\Bfootnote{creditur V n134}} ad credendum, igitur non potest probari, quia si posset evidenter probari, non exiret limites animae creatae \edtext{nec}{\Bfootnote{ac S V n135}} etiam investigationes humanae, et per consequens non esset articulus fidei, quia fides est de \edtext{his}{\Bfootnote{eis V n136}} quae excedunt facultatem humanam.
        \pend
      
        \bigskip
         
        \pstart
        \ledsidenote{\textbf{21}}
        Aliae sunt rationes quae tangunt alias materias quae reservabuntur loco suo, et istae sufficiant pro praesenti.
        \pend
       
        \bigskip
         \section*{[Divisio Quaestionis]} 
        \pstart
        \ledsidenote{\textbf{22}}
        Nunc autem restat aliqualiter declarare materiam pro cuius declaratione et \edtext{veritatis}{\Bfootnote{ V n137}} \edtext{circa}{\Bfootnote{hoc \textit{add. sed del.} V n138}} tertiam distinctionem quae ordine doctrinali debet esse prima, licet propter divisionem libri \name{Magistri}\index[persons]{Peter Lombard} primo loco non posuerit et bene secundum intentionem suam. Et circa igitur materiam \edtext{huius}{\Bfootnote{huiusmodi V n139}} tertiae distinctionis et inquisitionem veritatis erunt pro praesenti aliqui \ledsidenote{S36rb} articuli et erunt sex. \edtext{Primus}{\Bfootnote{Post S n140}} erit utrum de Deo possit haberi notitia incomplexa hic in via. Secundus erit de notitiarum intuitione et abstractione differentia. Tertius erit utrum aliquod praedicatum de Deo et creaturis univoce praedicetur. \edtext{Quartus}{\Bfootnote{Et quartus R SV n141}} erit utrum Deus reponatur sub aliquo decem praedicamentorum. Quintus \edtext{articulus}{\Bfootnote{\textit{om.} S V n142}} erit utrum Deum esse sit hic in via demonstrabile. Sextus erit de \edtext{responsionibus}{\Bfootnote{responsione S n143}} ad rationes iam factas.
        \pend
      
        \bigskip
         \section*{[Articulus Primus]} 
        \bigskip
         \section*{[Circa naturalis communicatio rerum]} 
        \pstart
        \ledsidenote{\textbf{23}}
        Quantum ad primum articulum resolvendo materiam ad divinam providentiam, quam \edtext{praemisi pro secundo principio theologico}{\Afootnote{Vide Lectio 18, "Secundum principium potest poni libera Dei omnipotentia..."}} omnia \edtext{debite}{\Bfootnote{determinate V n144}} gubernantem, sciendum est quod naturalis communicatio rerum est necessaria pro universi conservatione \ledsidenote{V40ra} maxime propter \edtext{animalia}{\Bfootnote{animantia V n145}} ut fugiant disconvenientia et convenientia prosequantur. Immo, secundum \edtext{\name{Magistrum}\index[persons]{Peter Lombard} II huius [libro],}{\Afootnote{Petrus Lombardus, \worktitle{Sententia}\index[works]{}, II, d. 1, c. 4, "Et sicut factus est homo propter Deum, id est, ut ei serviret, ita mundus factus est propter hominem, scilicet ut ei serviret."}} \edtext{et}{\Bfootnote{immo V n146}} secundum veritatem, omnia sunt facta propter hominem, immo etiam angeli sunt facti propter hominem, licet sint speciei superioris \edtext{et}{\Bfootnote{ant \textit{add. sed del.} R n147}} \edtext{nobilioris,}{\Bfootnote{perfectioris V \textit{om.} S n148}} caelum et terra, et omnis creatura, elementa propter mixta, mixta propter \edtext{animalia,}{\Bfootnote{animantia V n149}} et \edtext{animalia}{\Bfootnote{animantia V n150}} propter hominem.
        \pend
     
        \pstart
        \ledsidenote{\textbf{24}}
        Secundo \edtext{sciendum est}{\Bfootnote{notandum V n151}} quod res non possunt communicari immediate seipsis nec cognosci eo quod sensibile positum supra sensum non facit \edtext{sensationem,}{\Bfootnote{etiam \textit{in textu} V n152}} ut dicit  \name{Philosophus}\index[persons]{Aristotle} \edtext{III}{\Bfootnote{II R SV S n153}} \worktitle{De Anima}\index[works]{de Anima},  \edtext{ubi etiam habetur quod}{\Bfootnote{\textit{om.} V n154}} \edtext{\enquote{lapis non est in anima sed species lapidis.}}{\Afootnote{Aristoteles, \worktitle{De anima}\index[works]{}, III, c. 8 (431b29-432a1)}} \ledsidenote{SV214va} Ideo, divina providentia \edtext{ordinate}{\Bfootnote{ordinante V n155}} res habent adinvicem communicari multiplicando \ledsidenote{R30vb}  species et \edtext{quasi}{\Bfootnote{consequenter R n156}} seipsas, quantum possunt communicando, potentiis cognitivis ipsarum. Quia, ut tactum est, ut servetur debitus ordo, res adinvicem communicari oportet, et quia non possunt per se, ideo est instinctum eis quod se diffundant per suas species quantum possunt. Ideo, res sunt sui ipsius diffusivae \edtext{specialiter}{\Bfootnote{spiritualiter V n157}} causando in potentiis sensitivis ipsarum rerum cognitiones, et consequenter ascendendo ad intellectum \edtext{et}{\Bfootnote{ad \textit{in textu} V n158}} alias potentias interiores. Et etiam ex divina providentia potentiae provisum est de diversis organis, secundum diversitatem obiectorum, recipientibus species ipsorum, quia unus sensus mediante uno organo non posset de omnibus obiectis iudicare, sicut visus non iudicat de odoribus et saporibus. Consequenter, ordinavit unum sensum qui habeat iudicare de obiectis aliorum sensuum et ille est sensus communis. Et demum, ordinavit potentiam \edtext{reservativam,}{\Bfootnote{reservativam \textit{corr. ex} ordinativam SV n159}} deinde potentiam superiorem animae quae non indiget organo et ipsa est multiplex, non quin sit una et eadem res sed habet diversos actus. Nam ipsa inquantum est actualis speciebus vocatur \edtext{memorativa seu memoria}{\Bfootnote{memorativa S  memoria V n160}} et ipsa est \edtext{prima}{\Bfootnote{\textit{om.} V n161}} pars imaginis correspondens Patri in divinis qui vocatur memoria \ledsidenote{S36va} \edtext{fecunda.}{\Bfootnote{secunda R n162}} Ideo inquantum fecundatur per speciem intelligibilem, vocatur memoria. Inquantum productiva intellectionis, vocatur intellectus. Et inquantum \edtext{productiva volitionis,}{\Bfootnote{volitiva S n163}} vocatur voluntas. Consequenter sciendum est quod de intellectiva est ad propositum quod ipsa est multiplex. Ipsa primo est potentia simpliciter apprehensiva incomplexae, deinde compositiva, deinde \ledsidenote{V40rb} iudicativa, deinde assertiva, deinde abstractiva, deinde discursiva. Et secundum hoc \edtext{diversificantur}{\Bfootnote{diversificant V n164}} species notitiarum et \edtext{possent}{\Bfootnote{possunt V n165}} sic subdividi sicut potentiae.
        \pend
      
        \bigskip
         \section*{[Duae difficultates]} 
        \pstart
        \ledsidenote{\textbf{25}}
        Consequenter videndum \edtext{est}{\Bfootnote{quomodo obiectum sit sui ipsius \textit{add. sed del.} V n166}} unde consurget quod \edtext{aliqua}{\Bfootnote{\textit{om.} V n167}} notitia sui obiecti repraesentativa sit. Secundo videndum est quomodo obiectum sit \edtext{sui ipsius}{\Bfootnote{\textit{om.} S n168}} intentionaliter diffusivum. Ita quod \edtext{hic}{\Bfootnote{\textit{om.} V n169}} sunt duae difficultates. Una est, ex qua radice notitia dicitur sui obiecti esse repraesentativa, utrum hoc sit quia est similitudo sui obiecti vel ex natura specifica vel ex qua alia causa. Secunda est qualiter obiectum \edtext{posset}{\Bfootnote{potest V n170}} intentionaliter diffundere et \edtext{multiplicare}{\Bfootnote{multiplice S n171}} \edtext{sic}{\Bfootnote{\textit{om.} S n172}} species ipsius repraesentativas. Et primo, dicam unum verbum de secundo, et non finiam. \edtext{Secundo}{\Bfootnote{Et secundo V n173}} dicam de \edtext{primo}{\Bfootnote{secundo V n174}} et finiam in lectione sequenti, quia intendo aliqua dicere quae non reperio in scriptis.
        \pend
      
        \bigskip
         \section*{[Circa secundam difficultatem]} 
        \bigskip
         \section*{[Opinio Ioannis de Ripa]} 
        \pstart
        \ledsidenote{\textbf{26}}
        Quantum ad secundam, sciendum \edtext{est}{\Bfootnote{\textit{om.} V n175}} quod \edtext{\name{Magister Ioannis de Ripa}\index[persons]{John of Ripa} \ledsidenote{R31ra} respondet}{\Afootnote{Cf. Ioannes de Ripa, \worktitle{Prologi pars secunda}\index[works]{}, a. 2, concl. 1 (Combes II:366): "Nullum obiectum materiale potest ad aliquam qualitatem intellectualem active concurrere."}} quod obiectum immo \edtext{quod}{\Bfootnote{\textit{om.} V n176}} nullum obiectum materiale potest concurrere obiective ad causandum huiusmodi notitiam intuitivam in \edtext{potentia,}{\Bfootnote{notitia V n177}} cuius radix principalis est, quia, dicit \edtext{ipse,}{\Bfootnote{quod V n178}} nulla res agit ultra gradum proprium. \edtext{Ergo nulla \ledsidenote{SV214vb} res materialis agit aliquid immateriale}{\Bfootnote{unde sua prima ratio stat 
                          in hoc quod nullum obiectum materiale cuius \textit{in textu} V n179}} cuiusmodi sunt praedictae notitiae intuitivae, ut videtur et ad hoc facit tres rationes, sed mihi videtur quod omnes rationes suae fundantur in hoc quod nihil agit ultra gradum proprium.
        \pend
     
        \pstart
        \ledsidenote{\textbf{27}}
        Unde sua prima ratio stat in hoc quod nullum obiectum materiale potest immaterialiter agere, ergo non \edtext{potest}{\Bfootnote{\textit{om.} V n180}} producere huiusmodi \edtext{notitiam,}{\Bfootnote{notitiam \textit{corr. ex} speciem SV n181}} quia videtur esse immaterialis.
        \pend
     
        \pstart
        \ledsidenote{\textbf{28}}
        Secunda ratio: nihil agit perfectius \edtext{quam}{\Bfootnote{q \textit{add. sed del.} R n182}} sit ipsum agens. Modo quodlibet \edtext{immateriale,}{\Bfootnote{materiale R SV n183}} ut dicit, est perfectius materiali. 
        \pend
     
        \pstart
        \ledsidenote{\textbf{29}}
         \edtext{Tertio}{\Bfootnote{Secundo S  dicit \textit{in textu} V n184}} quia si sic, tunc obiecta materialia in sacramento altaris possent agere in organis corporis Christi ibidem existentis, calefaciendo, frigefaciendo, quod non \edtext{conceditur.}{\Bfootnote{concederetur. V n185}}
        \pend
     
        \pstart
        \ledsidenote{\textbf{30}}
         \edtext{Secunda conclusio quam ponit}{\Afootnote{Cf. Ioannis de Ripa, \worktitle{Prologi pars secunda}\index[works]{}, a. 2, concl. 2 (Combes II:368): "Non cuilibet obiecto immateriali potest correspondere activitas obiective."}} \edtext{quod}{\Bfootnote{quam S n186}} nec qualitas materialis agit intentionaliter producendo notitiam intuitivam sui ipsius, quia nihil agit ultra gradum proprium. Ipse enim incipit a superiori quia primo dicit quod non quaelibet substantia immaterialis potest concurrere ad sui notitiam obiective, et consequenter dicit quod \edtext{nec}{\Bfootnote{non V n187}} qualitas materialis. Et ut \edtext{brevius}{\Bfootnote{breviter V n188}} tangam \edtext{radicem}{\Bfootnote{rationem V n189}} suam, ipse supponit tria.
        \pend
     
        \pstart
        \ledsidenote{\textbf{31}}
         \edtext{Primum}{\Afootnote{Cf. Ioannis de Ripa, \worktitle{Prologi pars secunda}\index[works]{}, quaest. VI, art. 2, concl. 2, prop. 4 (Comes II:368).}} \ledsidenote{S36vb} est \ledsidenote{V40va} quod tota latitudo accidentium sit simpliciter \edtext{finita,}{\Bfootnote{finita \textit{corr. ex} infinita R n190}} ita quod tota latitudo specierum cognitionum \edtext{producibilium}{\Bfootnote{producibilium \textit{corr. ex} sensibilium SV n191}} \edtext{est}{\Bfootnote{sit S n192}} simpliciter finita in genere accidentium. \edtext{Secundo}{\Afootnote{Cf. Ioannis de Ripa, \worktitle{Prologi pars secunda}\index[works]{}, quaest. VI, art. 2, concl. 2, prop. 1 (Comes II:368).}} supponit quod semper obiecto alterius speciei correspondet notitia alterius speciei. \edtext{Et}{\Bfootnote{\textit{om.} R n193}} \edtext{tertio}{\Afootnote{Cf. Ioannis de Ripa, \worktitle{Prologi pars secunda}\index[works]{}, quaest. VI, art. 2, concl. 2, prop. 2 (Comes II:368).}} supponit quod species cognitionum sunt sicut numeri.
        \pend
     
        \pstart
        \ledsidenote{\textbf{32}}
        Istis \edtext{dictis,}{\Bfootnote{datis V n194}} imaginatur sic et capit obiectum infinitum quod causet notitiam sui obiective in potentia. Deinde \edtext{capiatur}{\Bfootnote{capitur V n195}} obiectum perfectius; causabit notitiam \edtext{nobiliorem.}{\Bfootnote{perfectiorem V n196}} \edtext{Et sic}{\Bfootnote{In V n197}} ascendendo et sic dabitur certus numerus. Et sic, data infima \edtext{notitia,}{\Bfootnote{\textit{om.} R SV S n198}} continue ascendendo, ex quo tota latitudo est \edtext{solum}{\Bfootnote{\textit{om.} S n199}} finita tandem \edtext{devenietur}{\Bfootnote{deveniretur V n200}} ad supremam, \edtext{et illa erit ultima quae possit ab obiecto in potentia obiective \edtext{causari,}{\Bfootnote{\textit{om.} S n202}} \edtext{etiam}{\Bfootnote{\textit{om.} S n203}} quia si obiectum excedens totam latitudinem obiectorum, quae possunt causare sui notitiam obiective, causaret in potentia notitiam sui obiective, illa esset alterius speciei et \edtext{superioris}{\Bfootnote{\textit{om.} S n204}} ad supremam}{\lemma{et \dots\ supremam}\Bfootnote{\textit{om.} R SV n201}} iam datam, quod non est dicendum.
        \pend
     
        \pstart
        \ledsidenote{\textbf{33}}
        Item, dato quod species non se \edtext{haberent}{\Bfootnote{haberet S n205}} sicut numeri, ex quo totum genus \edtext{accidentium}{\Bfootnote{accidentium \textit{add.} \textit{interl.} R n206}} est finitum, non posset \edtext{in}{\Bfootnote{\textit{om.} R n207}} infinitum procedi, et \edtext{sic}{\Bfootnote{\textit{om.} S n208}} tandem deveniretur ad supremam notitiam, vel \edtext{diceretur}{\Bfootnote{quod \textit{add. sed del.} SV n209}} quod proportionaliter sicut obiecta crescunt secundum perfectionem ita et notitiae, ergo etc.
        \pend
     
        \pstart
        \ledsidenote{\textbf{34}}
        Quantum \edtext{autem}{\Bfootnote{ergo V n210}} est de me ego \edtext{tenebo}{\Bfootnote{ego \textit{add. sed del.} V n211}} conclusionem contrariam respondendo ad istas rationes; et ad primum dubium motum, scilicet, an notitia ex sui natura sit sui obiecti repraesentativa, \edtext{et sic}{\Bfootnote{etiam V n212}} sua imaginatio prima habet satis magnam apparentiam, concessis suppositionibus, sed negando quod species \edtext{non}{\Bfootnote{\textit{om.} V n213}} se haberent \edtext{sicut}{\Bfootnote{sicut \textit{corr. ex} quod sint R n214}} numeri \edtext{illa}{\Bfootnote{secunda V n215}} imaginatio non \edtext{procedet,}{\Bfootnote{procederet V n216}} ergo etc.
        \pend
        
        \endnumbering
        
     
        \end{document}
    
