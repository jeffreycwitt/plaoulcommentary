
        %this tex file was auto produced from TEI by lombardpress-print on 2016-01-10T14:54:11.237-05:00 using the  file:/Users/JCWitt/Desktop/lombardpress-print/lbp-latex-critical.xsl 
        \documentclass[twoside, openright]{report}
        
        % etex package is added to fix bug with eledmac package and mac-tex 2015
        % See http://tex.stackexchange.com/questions/250615/error-when-compiling-with-tex-live-2015-eledmac-package
        \usepackage{etex}
        
        %imakeidx must be loaded beore eledmac
        \usepackage{imakeidx}
        
        \usepackage{eledmac}
        \usepackage{titlesec}
        \usepackage [english]{babel}
        \usepackage [autostyle, english = american]{csquotes}
        \usepackage{geometry}
        \usepackage{fancyhdr}
        \usepackage[letter, center, cam]{crop}
        
        
        \geometry{paperheight=10in, paperwidth=7in, hmarginratio=3:2, inner=1.7in, outer=1.13in, bmargin=1in} 
        
        %fancyheading settings
        \pagestyle{fancy}
        
        %git package 
        \usepackage{gitinfo2}
        
        %watermark
        \usepackage{draftwatermark}
        
        
        %quotes settings
        \MakeOuterQuote{"}
        
        %title settings
        \titleformat{\section} {\normalfont\scshape}{\thesection}{1em}{}
        \titlespacing\section{0pt}{12pt plus 4pt minus 2pt}{12pt plus 2pt minus 2pt}
        \titleformat{\chapter} {\normalfont\Large\uppercase}{\thechapter}{50pt}{}
        
        %eledmac settings
        \foottwocol{B}
        \linenummargin{outer}
        \sidenotemargin{inner}
        
        %other settings
        \linespread{1.1}
        
        %custom macros
        \newcommand{\name}[1]{\textsc{#1}}
        \newcommand{\worktitle}[1]{\textit{#1}}
        
        
        \begin{document}
        \fancyhead[RO]{Lectio 20, de Notitia [Sorbonne Transcription]}
        \fancyhead[LO]{Petrus Plaoul}
        \fancyhead[LE]{0.1.0+dev+\gitDescribe}
        \chapter*{Lectio 20, de Notitia [Sorbonne Transcription]}
        \addcontentsline{toc}{chapter}{Lectio 20, de Notitia [Sorbonne Transcription]}
        
         
        \beginnumbering
         \section*{Lectio 20, de Notitia [Sorbonne Transcription]} 
        \bigskip
         \section*{Recapitulatio prologi} 
        \pstart
        \ledsidenote{\textbf{1}}
        Sciendum est secundum quod in praecedenti lectione dicebatur in toto processu facto circa prologum visum est qualiter fides non subiacet humano iudicio et consequenter deductum est quod investigatio humana subiacet fidei et per eam dirigitur dictum est ulterius quod theologia participat fidei et investigationis humanae rationem ideo non participat fidei rationem ad concedendum proprie sed ad dirigendum et curandum humanam investigationem ab erroribus et falsis apparenciisapparentiis contra veritatem ideo fides sibi subdelegabit et dabitur sibi audientia ut possit inquirere veritatem
        \pend
      
        \bigskip
         \section*{Rationes Principales: quod Deus non est a nobis cognoscibilis} 
        \bigskip
         
        \pstart
        \ledsidenote{\textbf{2}}
        et primo circa primum principium positum scilicet \enquote*{deum esse} arguit investigatio humana et adducit testes secum contra fidem primo neganciumnegantium deum esse nam dixit incipiens  in corde suo non est deus alios adduxit \name{philosophos}\index[persons]{} dubitantes deum esse alios ponentes ipsum esse mere corporalem scilicet caelum et astra et huiusmodi
        \pend
     
        \pstart
        \ledsidenote{\textbf{3}}
        consequenter ex dictis fidei arguit quod deus non est a nobis cognoscibilis et quod sic testatur scriptura sacra deum nemo vidit umquam eciametiam \name{apostolus}\index[persons]{Paul, the Apostle} dicit quod inhabitat lucem inaccessibilem \name{dionysius}\index[persons]{} eciametiam dicit de ipso quod de ipso non est scientia nec fantasia nec opinio igitur non est a nobis cognoscibilis
        \pend
      
        \bigskip
         \section*{Prima Ratio} 
        \pstart
        \ledsidenote{\textbf{4}}
        deinde arguitur per rationes probando quod deus non est a nobis cognoscibilis quia si sic igitur cognitio dei a vel igitur ipsa est proportionata obiecto et tunc est infinita cum obiectum sit infinitum et sic anima humana non est ipsius capax vel est proportionata potentiae et sic est solum finita et improportionata obiecto et sic per consequens non erit notitia dei
        \pend
      
        \bigskip
         \section*{Secunda ratio} 
        \pstart
        \ledsidenote{\textbf{5}}
        2osecundo potest argui per comparationem loci ad locabile nam si esset aliquod mobile localiter distans infinite a suo loco in nullo tempore finito ymoimmo numquam posset moveri in illum locum nisi per formam infinitae virtutis modo loquendo proportionaliter de loco spirituali deus est locus spiritualis animarum infinite distans ab anima propter infinitam distantiam obiectorum igitur anima non poterit ipsum actingereattingere cognitive nisi per notitiam infinitam simpliciter quae est nobis impossibilis simpliciter
        \pend
      
        \bigskip
         \section*{Quarta ratio} 
        \pstart
        \ledsidenote{\textbf{6}}
        deinde probatur eciametiam quod nec adiutorio fidei potest anima sive luminis gloriae ad huiusmodi notitiam seu \ledsidenote{S35va}cognitionem ipsius dei elevari quia vel huiusmodi lumen esset divina essentia vel aliquid creatum sufficienter elevans potentiam animae cognitivam ad huiusmodi cognitionem dei non potest dici quod sit divina essentia quia tunc infinite immutaret animam
        \pend
     
        \pstart
        \ledsidenote{\textbf{7}}
        pro cuius declaratione supponitur quod quanto aliqua notitia est perfectioris speciei tanto perfectius immutat ita quod a priori perfectio notitiae actenditurattenditur penes hoc quod est perfectioris speciei et a posteriori perfectio speciei actenditurattenditur penes hoc quod est perfecciusperfectius immutare
        \pend
     
        \pstart
        \ledsidenote{\textbf{8}}
        modo considerata tota latitudine cognitionum creaturarum quae in infinitum procedit ascendendo secundum perfectionem specificam notitia ipsius dei est supra totam latitudinem notitiarum creatarum et est perfectioris speciei quam tota latitudo notitiarum creaturarum et tamen tota illa latitudo infinite immutat igitur notitia dei perfectiori modo quam infinite immutaret
        \pend
     
        \pstart
        \ledsidenote{\textbf{9}}
        si daretur Ista est forciorfortior ratio tenenciumtenentium quod deus non potest esse notitia creaturae et supleresupplere vicem speciei et volitionis et \name{magister Ioannes de Ripa}\index[persons]{John of Ripa} dicit quod deus supplens vicem speciei et volitionis est notitia infinita quo ad speciem et finita quo ad gradum
        \pend
     
        \pstart
        \ledsidenote{\textbf{10}}
        sed mihi videtur cum reverentia sua quod istud non sit multum bene dictum in materia quia eo ipso quod est infinita quo adquoad speciem est infinita simpliciter et sic infinita quo adquoad gradum sed non oporteret econverso si esset infinita quo adquoad gradum quod esset infinita simpliciter si enim esset infinita albedo quo ad gradum non propter hoc esset infinita simpliciter verbi gratia anima intellectiva est praecise finita et tamen in latitudine enciumentium esset perfectior quam infinita albedo si daretur ideo solutio non videtur vera quia sicut tactum est perfectio notitiae simpliciter magis accenditurattenditur quo ad speciem quam quo ad gradum modo per ipsum notitia divina suplenssupplens vicem speciei est infinita quo ad speciem igitur infinita simpliciter igitur infinite immutat nec potest dici quod lumen gloriae elevans animam sit aliquod creatum ex eo quod divina essentia est immensa et incomprehensibilis a creatura igitur non potest esse aliqua species vel notitia creata ipsam divinam essentiam secundum quodlibet sui repraesentans
        \pend
     
        \pstart
        \ledsidenote{\textbf{11}}
        Et in istis rationibus quandoque inducam de notitia beatifica quandoque de notitia hic in via et videbitur in solutione istarum rationum de utraque notitia
        \pend
      
        \bigskip
         \section*{Quinta ratio} 
        \pstart
        \ledsidenote{\textbf{12}}
        Item 5oquinto notitia talis esset simpliciter infinita istud patet proportionando notitias secundum proporcionemproportionem obiectorum nam generaliter perfectioris obiecti est perfectior cognitio et specialiter loquendo de \ledsidenote{S35vb}notitia intuitiva nam latitudo notitiarum creatarum est infinita scilicet continue ascendendo secundum notitiarum perfeccionemperfectionem quia quacumque data potest dari in dupploduplo perfectior in tripplotriplo et sic in infinitum sicut contingit de obiectis ipsarum noticiarumnotitiarum modo obiectum increatum si ipse deus est perfectior quam tota latitudo obiectorum creatorum si esset esset infinita igitur etc
        \pend
     
        \pstart
        \ledsidenote{\textbf{13}}
        et si dicatur quod non est eadem proportio cognitionum inter se sicut obiectorum ita quod non oportet quod obiecti perfectioris in dupploduplo sit cognitio perfectior in duplo retorquetur ideo iterum verbi gratia datis duabus noticiisnotitiis quarum una est duppladupla ad aliam si dicas quod ad causandam 2amsecundam non sufficit obiectum dupplaeduplae perfectionis si enim non sufficiat capiam obiectum incentuploin centuplo vel millecuplo perfectius quod sufficiet causare notitiam in duplo perfectiorem et sic augebo continue proportiones obiectorum intantum quod semper duplabitur notitia cum sit processus in infinitum in obiectis continue ascendendo secundum perfectionem specie et ita erit processus in infinitum in notitiis semper duplando notitias et in multis habet locum ista augmentatio proportionum
        \pend
      
        \bigskip
         \section*{Sexta ratio} 
        \pstart
        \ledsidenote{\textbf{14}}
        Item talis dei notitia esset perfectior quam esset una alia quae esset infinitorum obiectorum creatorum et cuiuslibet distincte sed illa esset infinita igitur et istam rationem facit \name{adam}\index[persons]{Adam Wodeham} quod autem notitia ipsius dei esset perfectior quam notitia infinitorum obiectorum creatorum et cuiuslibet distincte si essent patet quia deus est perfectius obiectum in se quam tota multitudo infinitorum obiectorum creatorum si esset
        \pend
      
        \bigskip
         \section*{Septima Ratio} 
        \pstart
        \ledsidenote{\textbf{15}}
        7oseptimo sic quia per talem cognitionem intuitivam deus apparet bonum finitum vel infinitum non potest dici quod apparet bonum finitum quia tunc aliqua creatura posset apparere intuitive melior deo et per consequenter diligibilior et consequenter sic videns posset rationaliter deo praeferre creaturam non potest dici quod apparet bonum infinitum quia tunc comprehenderetur quia cognosceretur secundum quodlibet sui
        \pend
      
        \bigskip
         \section*{Octava ratio} 
        \pstart
        \ledsidenote{\textbf{16}}
        6osexto potest argui per comparationem distantiae localis obiecti sensibilis a sensu nam propter magnam distantiam sensus decipitur circa obiectum suum et iudicat rem esse minorem quam sit ita proportionaliter de visu intellectuali propter infinitam distantiam decipitur circa obiectum suum infinite \ledsidenote{S36ra}ab eo distans et per consequens non possumus habere veram notitiam de deo
        \pend
     
        \pstart
        \ledsidenote{\textbf{17}}
        Item ex parte obiecti si esset aliquod luminosum infinitum quod infinite distaret a visu visus non iudicaret ipsum esse infinitum igitur ita erit de visu spirituali
        \pend
     
        \pstart
        \ledsidenote{\textbf{18}}
        Item eciametiam ex parte potentiae nam debilitas potentiae facit apparere obiectum remissius vel et minus quam sit igitur si potentia in infinitum distet ab obiecto in infinitum defficietdeficiet a veri obiecti cognitione istud patet de oculo male disposito ut caecutiente cui lux solis videtur remississima Istud eciametiam patet de oculo niticoracis unde propter debilitatem sui visus iudicat lumen solis remississimum intantum quod iudicat ipsum non sufficere ad videndum
        \pend
     
        \pstart
        \ledsidenote{\textbf{19}}
        Item dato quod deus esset ymaginabilisimaginabilis vel conceptibilis non sequitur quod sit sicut enim non sequitur vacuum est ymaginabileimaginabile vel chimera vel impossibile igitur vacuum est etc quia sicut dicit \name{philosophus}\index[persons]{Aristotle} 3o \worktitle{physicorum}\index[works]{} circa finem ymaginatumimaginatum non est credendum et per consequens dato quod primae rationes non concluderent quin deus esset ymaginabilisimaginabilis vel conceptibilis non propter hoc sequitur quod deus esset omnis enim ratio facta ad probandum quod deus sit est in apparenter solubilis sicut ipsa apparenter arguit ita quod considerata tota latitudine apparentiarum rationum probantium deum esse ex una parte et considerata tota latitudine apparentiarum solutionum illarum rationum ex alia parte latitudo apparentiarum in solutionibus esse aequalis latitudini apparentiarum rationum probantium deum esse et per consequens una pars non debet magis movere quam alia
        \pend
      
        \bigskip
         \section*{Nona ratio} 
        \pstart
        \ledsidenote{\textbf{20}}
        Item deum esse traditur ad credendum igitur non potest probari quia si posset probari evidenter non exiret limites animae creatae ac eciametiam investigationis humanae et per consequens non esset articulus fidei quia fides est de hiis quae excedunt facultatem humanam
        \pend
      
        \bigskip
         
        \pstart
        \ledsidenote{\textbf{21}}
        aliae sunt rationes quae tangunt alias materias quae reservabuntur loco suo et istae sufficiant pro praesenti
        \pend
       
        \bigskip
         \section*{Divisio Quaestionis} 
        \pstart
        \ledsidenote{\textbf{22}}
        nunc autem restat aliqualiter declarare materiam pro cuius declaratione et veritatis circa 3amtertiam distinctionem quae ordine doctrinali debet esse prima licet propter divisionem libri \name{magistri}\index[persons]{Peter Lombard} primo loco non posuerit et bene secundum intentionem suam Circa igitur materiam huius 3aetertiae distinctionis et inquisitionem veritatis erunt pro praesenti aliqui \ledsidenote{S36rb}articuli et erunt sex post erit utrum de deo possit haberi notitia incomplexa hic in via 2ussecundus erit de notitiarum intuitione et abstractione differentia 3ustertius erit utrum aliquod praedicatum de deo et creaturis univoce praedicetur 4usquartus erit utrum deus reponatur sub aliquo decem praedicamentorum 5usquintus erit utrum deum esse sit hic in via demonstrabile 6ussextus erit de Responsione ad rationes iam factas
        \pend
      
        \bigskip
         \section*{Articulus Primus} 
        \bigskip
         \section*{Circa naturalis communicatio rerum} 
        \pstart
        \ledsidenote{\textbf{23}}
        Quantum ad primum articulum resolvendo materiam ad divinam providentiam quam praemisi pro 2osecundo principio theologico omnia debite gubernantem sciendum est quod naturalis communicatio rerum est neccessarianecessaria pro universi conservatione maxime propter animalia ut fugiant disconvenientia et convenientia prosequantur ymoimmo secundum \name{magistrum}\index[persons]{} 2osecundo huius et secundum veritatem omnia sunt facta propter hominem caelum et terra et omnis creatura elementa propter mixta et mixta propter animalia animalia propter hominem ymoimmo eciametiam angeli facti sunt propter hominem licet sint speciei superioris et 
        \pend
     
        \pstart
        \ledsidenote{\textbf{24}}
        2osecundo sciendum est quod res non possunt immediate communicari se ipsis nec cognosci eo quod sensibile positum supra sensum non facit sensationem ut dicit \name{philosophus}\index[persons]{Aristotle} 2o \worktitle{de anima}\index[works]{de Anima} ubi eciametiam habetur quod lapis non est in anima sed species lapidis ideo divina providentia ordinate res habent adinvicem communicari multiplicando species et quasi se ipsas quantum possunt communicando potenciispotentiis cognitivis ipsarum quia ut tactum est ut servetur debitus ordo res ad invicem communicari oportet et quia non possunt per se ideo est instinctum eis quod se diffundant per suas species quantum possunt ideo res sunt sui ipsius diffusivae specialiter causando in potentiis sensitivis ipsarum rerum cognitiones et consequenter ascendendo ad intellectum et alias potentias interiores et eciametiam ex divina providentia potentiae provisum est de diversis organis secundum diversitatem obiectorum recipientibus species ipsorum quia unus sensus mediante uno organo non posset de omnibus obiectis iudicare sicut visus non iudicat de odoribus et saporibus consequenter ordinavit unum sensum qui habeat iudicare de obiectis aliorum sensuum et ille est sensus communis et demum ordinavit potentiam reservativam deinde potentiam superiorem animae quae non indiget organo et ipsa est multiplex non quin sit una et eadem res sed habet diversos actus nam ipsa inquantum est actualis speciebus vocatur memorativa et ipsa est prima pars ymaginisimaginis correspondens patri in divinis qui vocatur memoria \ledsidenote{S36va}fecunda ideo inquantum fecundatur per speciem intelligibilem vocatur memoria inquantum productiva intellectionis vocatur intellectus et inquantum productiva volitionis vocatur voluntas Consequenter sciendum est quia de intellectiva est ad propositum quod ipsa est multiplex ipsa primo est potentia simpliciter apprehensiva incomplexae deinde compositiva deinde Iudicativa deinde assertiva deinde abstractiva deinde discursiva et secundum hoc diversificantur species notitiarum et possent sic subdividi sicut potentiae
        \pend
      
        \bigskip
         \section*{[Duae difficultates]} 
        \pstart
        \ledsidenote{\textbf{25}}
        Consequenter videndum est unde consurgit quod aliqua notitia sit sui obiecti repraesentativa 2osecundo videndum est quomodo obiectum sit intentionaliter diffusivum ita quod hic sunt duae difficultates una est ex qua radice notitia dicitur sui obiecti esse repraesentativa utrum hoc sit quia est similitudo obiecti sui vel ex natura specifica vel ex qua alia causa 2asecunda est qualiter obiectum posset intentionaliter diffundere et multiplice species ipsius repraesentativas et primo dicam unum verbum de 2osecundo et non finiam 2osecundo dicam de primo et finiam in lectione sequenti quia intendo aliqua dicere quae non reperio in scriptis
        \pend
      
        \bigskip
         \section*{[Circa secundam difficultatem]} 
        \bigskip
         \section*{[Opinio Ioannis de Ripa]} 
        \pstart
        \ledsidenote{\textbf{26}}
        quantum ad 2msecundam sciendum est quod \name{magister Ioannes de Rippa}\index[persons]{John of Ripa} Respondet quod obiectum ymoimmo quod nullum obiectum materiale potest concurrere obiective ad causandum huiusmodi noticiamnotitiam intuitivam in potentia cuius radix principalis est quia dicit ipse nulla res agit ultra gradum proprium igitur nulla res materialis agit aliquid immateriale cuiusmodi sunt praedictae notitiae intuitivae ut videtur et ad hoc facit tres rationes sed mihi videtur quod omnes suae rationes fundantur in hoc quod nihil agit ultra gradum proprium
        \pend
     
        \pstart
        \ledsidenote{\textbf{27}}
        unde sua prima ratio stat in hoc quod nullum obiectum materiale potest immaterialiter agere igitur non potest producere huiusmodi notitiam quia videtur esse immaterialis
        \pend
     
        \pstart
        \ledsidenote{\textbf{28}}
        2asecunda ratio nihil agit perfectius quam sit ipsum agens modo quodlibet immateriale ut dicit est perfectius materiali
        \pend
     
        \pstart
        \ledsidenote{\textbf{29}}
        2osecundo quia si sic tunc obiecta materialia in sacramento altaris possent agere in organis corporis christi ibidem existentis calefaciendo frigefaciendo quod non conceditur
        \pend
     
        \pstart
        \ledsidenote{\textbf{30}}
        2asecunda conclusio quam ponit quam nec qualitas materialis agit intentionaliter producendo notitiam intuitivam sui ipsius quia nihil agit ultra gradum proprium ipse enim incipit a superiori quia primo  dicit quod non quaelibet substantia immaterialis potest concurrere ad sui notitiam obiective et consequenter dicit quod nec qualitas materialis et ut brevius tangam radicem suam ipse supponit 3atria
        \pend
     
        \pstart
        \ledsidenote{\textbf{31}}
        primum \ledsidenote{S36vb}est quod tota latitudo accidentium sit simpliciter finita ita quod tota latitudo specierum cognitionum producibilium sit simpliciter finita in genere accidentium 2osecundo supponit quod semper obiecto alterius speciei correspondet notitia alterius speciei et 3otertio supponit quod species cognitionum sunt sicut numeri
        \pend
     
        \pstart
        \ledsidenote{\textbf{32}}
        Istis dictis ymaginaturimaginatur sic et capit obiectum infinitum quod causet notitiam sui obiective in potentia deinde capiatur obiectum perfectius causabit notitiam nobiliorem et sic ascendendo et sic dabitur certus numerus et sic data infima continue ascendendo ex quo tota latitudo est finita tandem devenietur ad supremam et illa erit ultima quae possit in potentia ab obiecto obiective quia si obiectum excedens totam latitudinem obiectorum quae possunt creare sui notitiam obiective causaret notitiam sui in potentia obiective illa esset alterius speciei et ad supremam iam datam quod non est dicendum
        \pend
     
        \pstart
        \ledsidenote{\textbf{33}}
        Item dato quod species non se haberet sicut numeri ex quo totum genus accidentium est finitum non posset in infinitum procedi et tandem deveniretur ad supremam notitiam vel diceretur quod proportionaliter sicut obiecta crescunt secundum perfectionem ita et notitiae ideo etc.
        \pend
     
        \pstart
        \ledsidenote{\textbf{34}}
        quantum autem est de me ego tenebo conclusionem contrariam Respondendo ad istas rationes et ad primum dubium motum scilicet an notitia ex sui natura sit sui obiecti repraesentativa et sic sua ymaginatioimaginatio prima habet satis magnam apparentiam concessis suppositionibus sed negando quod species non se haberent sicut numeri illa ymaginatioimaginatio non procedet ideo etc.
        \pend
        
        \endnumbering
        
     
        \end{document}
    
