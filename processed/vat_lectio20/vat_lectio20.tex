
        %this tex file was auto produced from TEI by lombardpress-print on 2016-01-10T14:54:21.365-05:00 using the  file:/Users/JCWitt/Desktop/lombardpress-print/lbp-latex-critical.xsl 
        \documentclass[twoside, openright]{report}
        
        % etex package is added to fix bug with eledmac package and mac-tex 2015
        % See http://tex.stackexchange.com/questions/250615/error-when-compiling-with-tex-live-2015-eledmac-package
        \usepackage{etex}
        
        %imakeidx must be loaded beore eledmac
        \usepackage{imakeidx}
        
        \usepackage{eledmac}
        \usepackage{titlesec}
        \usepackage [english]{babel}
        \usepackage [autostyle, english = american]{csquotes}
        \usepackage{geometry}
        \usepackage{fancyhdr}
        \usepackage[letter, center, cam]{crop}
        
        
        \geometry{paperheight=10in, paperwidth=7in, hmarginratio=3:2, inner=1.7in, outer=1.13in, bmargin=1in} 
        
        %fancyheading settings
        \pagestyle{fancy}
        
        %git package 
        \usepackage{gitinfo2}
        
        %watermark
        \usepackage{draftwatermark}
        
        
        %quotes settings
        \MakeOuterQuote{"}
        
        %title settings
        \titleformat{\section} {\normalfont\scshape}{\thesection}{1em}{}
        \titlespacing\section{0pt}{12pt plus 4pt minus 2pt}{12pt plus 2pt minus 2pt}
        \titleformat{\chapter} {\normalfont\Large\uppercase}{\thechapter}{50pt}{}
        
        %eledmac settings
        \foottwocol{B}
        \linenummargin{outer}
        \sidenotemargin{inner}
        
        %other settings
        \linespread{1.1}
        
        %custom macros
        \newcommand{\name}[1]{\textsc{#1}}
        \newcommand{\worktitle}[1]{\textit{#1}}
        
        
        \begin{document}
        \fancyhead[RO]{Lectio 20, de Notitia [Vatican Transcription]}
        \fancyhead[LO]{Petrus Plaoul}
        \fancyhead[LE]{0.1.0+dev+\gitDescribe}
        \chapter*{Lectio 20, de Notitia [Vatican Transcription]}
        \addcontentsline{toc}{chapter}{Lectio 20, de Notitia [Vatican Transcription]}
        
         
        \beginnumbering
         \section*{Lectio 20, de Notitia [Vatican Transcription]} 
        \bigskip
         \section*{Recapitulatio prologi} 
        \pstart
        \ledsidenote{\textbf{1}}
        Sciendum est secundum quod in praecedenti lectione dicebatur in toto processu facto circa prologum visum est qualiter fides non subiacet humano iudicio. et consequenter deductum est quod investigatio humana subiacet fidei et per eam dirigitur dictum est quod theologia participat fide et investigationis humanae ideo non participat fidei ad concedendum proprie sed ad dirigendum et curandum humanam investigationem fidei ab erroribus et falsis apparentiis contra veritatem ideo fides sibi subdelegabit et dabitur sibi audientia ut possit inquirere veritatem
        \pend
      
        \bigskip
         \section*{Rationes Principales: quod Deus non est a nobis cognoscibilis} 
        \bigskip
         
        \pstart
        \ledsidenote{\textbf{2}}
        et primo circa primum principium scilicet \enquote*{deum esse} arguit investigatio humana et adducit testes sicut contra fidem primo negantium deum etiam alios ponentes ipsum esse mere corporalem scilicet caelum astra et huiusmodi 
        \pend
     
        \pstart
        \ledsidenote{\textbf{3}}
        consequenter ex dictis fidei arguit quod deus non est a nobis cognoscibilis \ledsidenote{V39ra}et quod ita sic testatur sacra scriptura quia scriptum est deum nemo vidit umquam \name{apostolus}\index[persons]{Paul, the Apostle} etiam dicit quod deus inhabitat lucem inaccessabileminaccessibilem  \name{dionysius}\index[persons]{Pseudo-Dionysius} etiam dicit quod de ipso non est scientia nec oppinioopinio nec fantasiaphantasia ergo non est a nobis cognoscibilis
        \pend
      
        \bigskip
         \section*{Prima Ratio} 
        \pstart
        \ledsidenote{\textbf{4}}
        deinde arguitur per rationes probando quod deus non est a nobis cognoscibilis quia si sic sit ergo cognitio dei a tunc ipsa est proportionata obiecto et est infinita et sic anima humana non est ipsius capax vel est proportionata potentiae et sic est solum finita et improportionata obiecto et per consequens non erit notitia dei
        \pend
      
        \bigskip
         \section*{Secunda ratio} 
        \pstart
        \ledsidenote{\textbf{5}}
        2osecundo potest argui per comparationem loci ad locale nam si esset aliquod mobile localiter distans infinite a suo loco in nullo tempore finito ymmoimmo numquam posset moveri in illum locum nisi per formam infinitae virtutis modo loquendo proportionaliter de loco spirituali deus est locus spiritualis animarum infinite distans ab anima propter infinitam distantiam eorum ergo anima non potest eum attingere cognitive nisi per notitiam infinitam simpliciter quae est nobis impossibilis hic in via
        \pend
       
        \bigskip
         \section*{Tertia ratio} 
        \pstart
        \ledsidenote{\textbf{6}}
        Vide Lectio 30
        \pend
       
        \bigskip
         \section*{Quarta ratio} 
        \pstart
        \ledsidenote{\textbf{7}}
        deinde probatur quod nec etiam adiutorio luminis gloriae possit anima ad huiusmodi notitiam vel cognitionem ipsius dei elevari quia vel huiusmodi lumen esset divina essentia vel aliquid creatum sufficienter elevans potentiae animae cognitivam ad huiusmodi cognitionem dei non potest dici quod sit divina essentia quia tunc infinite immutaret animam
        \pend
     
        \pstart
        \ledsidenote{\textbf{8}}
        pro cuius declaratione, supponitur quod quanto aliqua notitia est perfectioris speciei tanto perfectius immutat illud patet clare quasi inductive fundatur etiam Ratione quia perfectio noticiaenotitiae sumitur penes hoc quod perfectori modo immutat ita quod a posteriori perfectio notitiae attenditur penes hoc quod est perfectioris speciei et a posteriori perfectio speciei attenditur penes perfectius immutare
        \pend
     
        \pstart
        \ledsidenote{\textbf{9}}
        modo considerata tota latitudine cognitionum creaturarum quae in infinitum procedit ascendendo semper secundum perfectionem specificam notitia ipsius dei est supra creaturamcreatam latitudinem notitiarum creatarum et tamen non tota illa latitudo infinite in mutatimmutat ergo notitia dei perfectiori modo quam infinite immutaret
        \pend
     
        \pstart
        \ledsidenote{\textbf{10}}
        si daretur et ista est \ledsidenote{V39rb}fortior ratio in materia tenentium quod deus potest esse notitia creaturae et supplere vicem speciei et volitionis et \name{magister ioannes de Ripa}\index[persons]{John of Ripa} dicit quod deus supplens vicem speciei et volitionis et est notitia infinita quo adquoad speciem et finita quo adquoad gradum
        \pend
     
        \pstart
        \ledsidenote{\textbf{11}}
        Sed mihi videtur cum sui reverentia quod illud non sit bene dictum quia eo ipso quod est infinita secundum speciem est infinita simpliciter et sic infinita quo adquoad gradum sed non oportet econverso si esset infinita quo adquoad gradum verbi gratia: anima intellectiva est praecise finita et tamen in latitudine entium esset perfectior quam infinita albedo si daretur et ideo solutio non videtur nam quia sicut tactum est perfectio notitiae simpliciter magis attenditur quo adquoad speciem quam quo adquoad gradum modo per ipsum notitia divina supplens vicem speciei etc est infinita quo adquoad speciem ergo infinita simpliciter  ergo infinite immutat nec potest dici quod lumen gloriae elevans animam etc sit aliquod creatum ex eo essentia divina est immensa et incomprehensibilis a creatura ergo non potest esse aliqua notitia creata vel species ipsam essentiam divinam secundum quodlibet simpliciter  repraesentans
        \pend
     
        \pstart
        \ledsidenote{\textbf{12}}
        et in istis rationibus quandoque inducam de notitia beatifica quandoque de notitia hic in via et videbitur in solutionibus istarum rationum de utraque noticianotitia
        \pend
      
        \bigskip
         \section*{Quinta ratio} 
        \pstart
        \ledsidenote{\textbf{13}}
        Item 5oquinto talis noticianotitia esset infinita simpliciter illud patet proportionando noticiasnotitias secundum proportionem obiectorum nam generaliter perfectioris obiecti est perfectior cognitio et specialiter loquendo de noticianotitia intuitiva modo latitudo notitiarum creatarum est infinita continue scilicet ascendendo secundum perfectionem notitiarum quia quacumque data potest dari in 2loduplo perfectior in 3lotriplo et sic in infinitum, sic contingit de obiectis ipsarum notitiarum modo obiectum increatum scilicet ipse deus est perfectius quam tota latitudo infinita obiectorum creatorum esset ergo notitia ipsius dei erit perfectior quam notitia infinitorum obiectorum creatorum et si esset infinita ergo
        \pend
     
        \pstart
        \ledsidenote{\textbf{14}}
        si dicatur duae non est eadem proportio cognitionum inter se, sicudsicut obiectorum ita quod non semper oportet quod obiecti perfectioris \ledsidenote{V39va}In duplo sit noticianotitia perfectior in duplo retorquetur ratio iterum verbi gratia datis duabus noticiisnotitiis quarum una est dupla ad aliam si dicis quod eandem secundam non sufficit obiectum 2laeduplae perfectionis si enim non sufficiat capiam obiectum in centuplo vel millecuplo perfectius quod sufficiet causare noticiamnotitiam in 2loduplo perfectiorem et sic augebo continue proportiones obiectorum intantum quod semper dupla videtur notitia cum sit processus in infinitum in noticiisnotitiis semper duplicando noticiasnotitias et in multis habet locum ista augmentatio proportionum
        \pend
      
        \bigskip
         \section*{Sexta ratio} 
        \pstart
        \ledsidenote{\textbf{15}}
        Item talis dei noticianotitia esset perfectior quam esset una alia infinitorum obiectorum creatorum et cuiuslibet distincte si ipsa esset infinita ergo et istam rationem facit \name{adam}\index[persons]{Adam Wodeham} antecedens patet quod notitia dei esset perfectior etc quia deus est perfectius obiectum in se quam tota latitudo obiectorum creatorum si esset 
        \pend
      
        \bigskip
         \section*{Septima Ratio} 
        \pstart
        \ledsidenote{\textbf{16}}
         7oseptimo quia vel per talem cognitionem intuitivam deus appareret bonum finitum vel infinitum non primum quia tunc aliqua creatura posset apparere intuitive melior deo et consequenter diligibilior et consequenter sic videns posset rationabiliter deo praeferre creaturam nec potest dici quod appareret bonum infinitum quia si sic tunc comprehenderetur quia cognosceretur secundum quodlibet sui
        \pend
      
        \bigskip
         \section*{Octava ratio} 
        \pstart
        \ledsidenote{\textbf{17}}
        8ooctavo potest argui per comparationem distantiae localis obiecti sensibilis a sensu nam propter materiam distantiam sensus decipitur circa suum proprium obiectum et iudicat rem esse minorem quam sit ita proportionaliter de visu intellectuali propter infinitam distantiam decipitur circa obiectum suum infinite ab eo distans et per consequens non possumus habere veram noticiamnotitiam de ipso deo
        \pend
     
        \pstart
        \ledsidenote{\textbf{18}}
        Item ex parte obiecti si esset aliquod luminosum infinitum quod infinite distaret a visu visus non iudicaret \enquote*{ipsum esse infinitum} ergo ita erit de visu spirituali
        \pend
     
        \pstart
        \ledsidenote{\textbf{19}}
        Item etiam ex parte nam debilitas potentiae facit apparere obiectum remissius et minus quam sit ergo si potentia in infinitum distet ab obiecto in infinitum deficiet a natura obiecti cognitione illud patet de oculo male disposito \ledsidenote{V39vb}ut in caecitate lux solis videtur remississima illud etiam apparet in oculo niticoracis unde propter debilitatem sui visus iudicat lumen solis remississimum intantum quod iudicat ipsum non sufficere ad videndum
        \pend
     
        \pstart
        \ledsidenote{\textbf{20}}
        Item dato quod deus esset ymaginabilisimaginabilis vel conceptibilis non propter hoc sequitur quod deus esset omnis enim ratio ad probandum \enquote*{quod Deus sit} est ita apparenter solubilis sicut ipsa arguit apparenter ita quod considerata latitudine apparentiarum rationum probabilis \enquote*{deum esse} ex una parte et considerata latitudine solutionum apparentiarum illarum rationum ex alia parte latitudo apparentiarum rationum probantium  \enquote*{deum esse} et per consequens una pars non debet magis movere quam alia
        \pend
      
        \bigskip
         \section*{Nona ratio} 
        \pstart
        \ledsidenote{\textbf{21}}
        Item \enquote*{deum esse} creditur ad credendum ergo non potest probari quia si posset evidenter probari non exiret limites animae creatae ac etiam investigationis humanae et per consequens non esset articulus fidei quia fides est de eis quae excedunt facultatem humanam
        \pend
      
        \bigskip
         
        \pstart
        \ledsidenote{\textbf{22}}
        aliae sunt rationes quae tangunt alias materias quae reservabuntur suo loco et istae sufficiant pro praesenti
        \pend
       
        \bigskip
         \section*{Divisio Quaestionis} 
        \pstart
        \ledsidenote{\textbf{23}}
        nunc autem restat aliqualiter declarare materiam pro cuius declaratione et veritate circa 3amtertiam distinctionem quae ordine doctrinali debet esse prima licet propter divisionem libri \name{magistri}\index[persons]{Peter Lombard} non primo loco ponit et bene secundum intentionem suam et circa hoc igitur materiam huiusmodi 3aetertiae distinctionis et inquisitionem veritatis erunt pro praesenti aliqui articuli et erunt sex primus erit utrum de deo possit haberi notitia incomplexa hic in via 2ussecundus erit de noticiarumnotitiarum intuitione et abstractione differentia 3ustertius erit utrum aliquod praedicatum de deo et creaturis praedicetur univoce 4usquartus erit utrum deus reponatur sub aliquo decem praedicamentorum 5usquintus erit utrum deum esse sit hic in via demonstrabile 6ussextus erit de responsionibus ad rationes iam factas 
        \pend
      
        \bigskip
         \section*{Articulus Primus} 
        \bigskip
         \section*{Circa naturalis communicatio rerum} 
        \pstart
        \ledsidenote{\textbf{24}}
        Quantum ad primum articulum resolvendo materiam ad divinam providentiam quam praemisi pro 2osecundo principio theologico omnia determinate gubernantem sciendum est quod naturalis communicatio rerum est neccessarianecessaria pro universi conservatione \ledsidenote{V40ra}maxime propter animantia ut fugiant disconvenientia et prosequantur convenientia ymmoimmo secundum \name{magistrum}\index[persons]{Peter Lombard} 2o huius ymmoimmo secundum veritatem omnia sunt facta propter hominem caelum et terra et omnis creatura elementa scilicet propter mixta et mixta propter animantia et animantia propter hominem ymoimmo et angeli facti sunt propter hominem licet sint superioris speciei et perfectioris
        \pend
     
        \pstart
        \ledsidenote{\textbf{25}}
        2osecundo notandum quod res non possunt communicari immediate seipsis nec cognosci eo quod sensibile positum supra sensum non facit sensationem etiam ut dicit \name{philosophus}\index[persons]{Aristotle} 3o \worktitle{de anima}\index[works]{de Anima} lapis non est in anima sed species lapidis ideo divina providentia ordinante res habent adinvicem communicari multiplicando species et quasi seipsas quantum possunt communicando potentiis cognitivis ipsarum quia ut tactum est ut servetur debitus ordo oportet res adinvicem communicari et quia non possunt per se ideo est instinctum  eis quod se diffundant per suas species quantum possunt ideo res sunt sui ipsius diffusivae spiritualiter causando in potentiis sensitivis ipsarum rerum cognitiones et consequenter ascendendo ad intellectum et ad alias potentias interiores et etiam ex divina providentia potentiae provisum est de diversis organis secundum diversitatem obiectorum recipientibus species ipsorum quia unus sensus mediante uno organo non posset de omnibus obiectis iudicare sicut visus non iudicat de odoribus et saporibus etc consequenter ordinavit unum sensum qui habeat iudicare de obiectis aliorum sensuum et ille est sensus communis et demum ordinavit potentiam reservativam deinde potentiam superiorem animae quae non indiget organo et ista est multiplex non quin sit una et eadem res sed habet diversos actus nam ipsa inquantum est actualis speciebus vocatur memoria et ipsa est pars ymaginisimaginis correspondens patri in divinis qui vocatur memoria fecunda ideo inquantum fecundatur per speciem intelligibilem vocatur memoria inquantum productiva intellectionis vocatur intellectus et inquantum volitiva vocatur voluntas   Consequenter sciendum est quod de intellectiva est ad propositum quod ipsa est multiplex ipsa primo est potentia simpliciter apprehensiva incomplexae deinde compositiva deinde \ledsidenote{V40rb}iudicativa deinde assertiva deinde abstractiva deinde discursiva et secundum hoc diversificant species notitiarum et possunt sic subdividi sicudsicut potentiae
        \pend
      
        \bigskip
         \section*{[Duae difficultates]} 
        \pstart
        \ledsidenote{\textbf{26}}
        Consequenter videndum est quomodo obiectum sit sui ipsius unde consurgit quod notitia sui obiecti repraesentativa sit 2osecundo videndum est quomodo obiectum sit sui ipsius intentionaliter diffusivum ita quod sunt duae difficultates una est ex qua radice notitia dicitur esse sui obiecti repraesentativa utrum hoc sit quia est similitudo sui obiecti vel ex natura specifica vel ex qua alia causa  2asecunda est qualiter obiectum potest intentionaliter diffundere et multiplicare sic species ipsius repraesentativas et primo dicam unum verbum de 2osecundo et non finiam et 2osecundo dicam de 2osecundo et finiam in lectione sequenti quia intendo aliqua dicere quae non reperio in scriptis
        \pend
      
        \bigskip
         \section*{[Circa secundam difficultatem]} 
        \bigskip
         \section*{[Opinio Ioannis de Ripa]} 
        \pstart
        \ledsidenote{\textbf{27}}
        Quantum ad 2msecundam sciendum quod \name{magister Ioannes de Ripa}\index[persons]{John of Ripa} respondet quod obiectum ymoimmo nullum obiectum materiale potest concurrere obiective ad causandum huiusmodi noticiamnotitiam intuitivam in notitia cuius Radix principalis est quia dicit quod nulla res agit ultra gradum proprium unde sua prima ratio stat in hoc quod nullum obiectum materiale cuius sunt praedictae notitiae intuitivae ut videtur et facit ad hoc tres rationes sed mihi videtur quod omnes rationes suae fundantur in hoc quod nichilnihil agit ultra gradum proprium
        \pend
     
        \pstart
        \ledsidenote{\textbf{28}}
        unde sua prima ratio stat in hoc quod nullum obiectum materiale potest immaterialiter agere ergo non producere huiusmodi noticiamnotitiam quia videtur esse immaterialis
        \pend
     
        \pstart
        \ledsidenote{\textbf{29}}
        2asecunda ratio nichilnihil agit perfectius quam sit ipsum agens modo quodlibet immateriale est perfectius materiali ut dicit ergo
        \pend
     
        \pstart
        \ledsidenote{\textbf{30}}
        3otertio sic quod si sic tunc obiecta materialia in sacramento altaris possent agere in organis corporis christi ibidem existentis calefaciendo frigefaciendo etc quod non concederetur
        \pend
     
        \pstart
        \ledsidenote{\textbf{31}}
        2asecunda conclusio quam ponit quod nec qualitas materialis agit intentionaliter producendo noticiamnotitiam intuitivam sui ipsius quia nichilnihil agit ultra gradum proprium ipse enim incipit a superiori quia primo dicit quod non quaelibet substantia immaterialis potest concurrere ad sui noticiamnotitiam obiective et consequenter dicit quod non qualitas materialis et ut breviter tangam rationem suam ipse supponit tria
        \pend
     
        \pstart
        \ledsidenote{\textbf{32}}
        primum est \ledsidenote{V40va}quod tota latitudo accidentium sit simpliciter finita ita quod tota latitudo specierum cognitionum producibilium est simpliciter finita in genere accidentium 2osecundo supponit quod semper obiecto alterius speciei correspondet noticianotitia alterius speciei et 3otertio supponit quod species cognitionum sunt sicut numeri
        \pend
     
        \pstart
        \ledsidenote{\textbf{33}}
        istis datis ymaginaturimaginatur sic et capit obiectum infinitum quod causet noticiamnotitiam sui obiective in potentia deinde capitur obiectum perfectius causabit noticiamnotitiam perfectiorem in ascendendo et sic dabitur certus numerus et sic data infima noticianotitia continue ascendendo ex quo tota latitudo solum est finita tandem deveniretur ad supremam et illa erit ultima quae possit ab obiecto in potentia obiective causari etiam quia si obiectum excedens totam latitudinem obiectorum quae possunt causare sui noticiamnotitiam obiective causaret in potentia noticiamnotitiam sui obiective illa esset alterius speciei et superioris ad supremam iam datam quod non est dicendum
        \pend
     
        \pstart
        \ledsidenote{\textbf{34}}
        Item dato quod species non se haberent sicut numeri ex quo totum genus accidentium est finitum non posset in infinitum procedi et sic tandem deveniretur ad supremam noticiamnotitiam etc vel diceretur quod proportionaliter sicut obiecta crescunt secundum perfectionem ita et notitiae etc
        \pend
     
        \pstart
        \ledsidenote{\textbf{35}}
        quantum ergo est de me ego tenebo ego conclusionem contrariam  respondendo ad istas rationes et ad primum dubium motum scilicet an notitia ex sui natura sit sui obiecti repraesentativa etiam sua ymaginatioimaginatio prima habet satis magnam apparentiam concessis suppositionibus sed negando quod species se haberent sicut numeri  2asecunda ymaginatioimaginatio non procederet ideo etc
        \pend
        
        \endnumbering
        
     
        \end{document}
    
