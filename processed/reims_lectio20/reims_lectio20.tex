
        %this tex file was auto produced from TEI by lombardpress-print on 2016-01-10T14:54:06.407-05:00 using the  file:/Users/JCWitt/Desktop/lombardpress-print/lbp-latex-critical.xsl 
        \documentclass[twoside, openright]{report}
        
        % etex package is added to fix bug with eledmac package and mac-tex 2015
        % See http://tex.stackexchange.com/questions/250615/error-when-compiling-with-tex-live-2015-eledmac-package
        \usepackage{etex}
        
        %imakeidx must be loaded beore eledmac
        \usepackage{imakeidx}
        
        \usepackage{eledmac}
        \usepackage{titlesec}
        \usepackage [english]{babel}
        \usepackage [autostyle, english = american]{csquotes}
        \usepackage{geometry}
        \usepackage{fancyhdr}
        \usepackage[letter, center, cam]{crop}
        
        
        \geometry{paperheight=10in, paperwidth=7in, hmarginratio=3:2, inner=1.7in, outer=1.13in, bmargin=1in} 
        
        %fancyheading settings
        \pagestyle{fancy}
        
        %git package 
        \usepackage{gitinfo2}
        
        %watermark
        \usepackage{draftwatermark}
        
        
        %quotes settings
        \MakeOuterQuote{"}
        
        %title settings
        \titleformat{\section} {\normalfont\scshape}{\thesection}{1em}{}
        \titlespacing\section{0pt}{12pt plus 4pt minus 2pt}{12pt plus 2pt minus 2pt}
        \titleformat{\chapter} {\normalfont\Large\uppercase}{\thechapter}{50pt}{}
        
        %eledmac settings
        \foottwocol{B}
        \linenummargin{outer}
        \sidenotemargin{inner}
        
        %other settings
        \linespread{1.1}
        
        %custom macros
        \newcommand{\name}[1]{\textsc{#1}}
        \newcommand{\worktitle}[1]{\textit{#1}}
        
        
        \begin{document}
        \fancyhead[RO]{Lectio 20, de Notitia [Reims Transcription]}
        \fancyhead[LO]{Petrus Plaoul}
        \fancyhead[LE]{0.1.0+dev+\gitDescribe}
        \chapter*{Lectio 20, de Notitia [Reims Transcription]}
        \addcontentsline{toc}{chapter}{Lectio 20, de Notitia [Reims Transcription]}
        
         
        \beginnumbering
         \section*{Lectio 20, de Notitia [Reims Transcription]} 
        \bigskip
         \section*{Recapitulatio prologi} 
        \pstart
        \ledsidenote{\textbf{1}}
        Sciendum est secundum quod in praecedenti lectione dicebatur in toto processu facto circa prologum visum est qualiter fides non subiacet humano Iudicio  et consequenter deductum est quod investigatio humana subiacet fidei et per eam dirigitur dictum est ulterius quod theologia participat fidei et investigationis humanae rationem ideo non participat fidei rationem ad concedendum proprie sed ad dirigendum et curandum humanam investigationem ab erroribus et falsis apparentiis contra veritatem ideo fides sibi subdelegabit et dabitur sibi audientia ut posset inquirere veritatem
        \pend
      
        \bigskip
         \section*{Rationes Principales: quod Deus non est a nobis cognoscibilis} 
        \bigskip
         
        \pstart
        \ledsidenote{\textbf{2}}
        Et primo circa primum principium positum scilicet \enquote*{deum esse} arguit investigatio humana et adducit testes secum contra fidem primo negantium \enquote*{deum esse} nam dicit insipiens in corde suo \enquote*{non est deus} et alios adducit philosophos dubitantes \enquote*{deum esse}  alios ponentes ipsum esse mere corporalem scilicet caelum et astra et huiusmodi
        \pend
     
        \pstart
        \ledsidenote{\textbf{3}}
        consequenter ex dictis fidei arguit quod deus non est a nobis cognoscibilis et quod ita sic testatur scriptura sacra deum nemo vidit umquam etiam \name{apostolus}\index[persons]{Paul, the Apostle} dicit quod inhabitat lucem in accessibileminaccessibilem  \name{Dionysius}\index[persons]{Pseudo-Dionysius} \ledsidenote{R29vb}etiam dicit de ipso quod de ipso non est scientia nec opinio nec fantasiaphantasia igitur non est a nobis cognoscibilis
        \pend
      
        \bigskip
         \section*{Prima Ratio} 
        \pstart
        \ledsidenote{\textbf{4}}
        Deinde arguitur per rationes probando quod deus non est a nobis cognoscibilis quia si sic sit ergo cognitio ipsius dei a vel igitur ipsa est proportionata obiecto et tunc est infinitum  cum obiectum est infinitum et sic anima humana non est ipsius capax  vel est proportionata potentiae et sic solum est finita et in proportionataimproportionata obiecto et per consequens non erit notitia dei
        \pend
      
        \bigskip
         \section*{Secunda ratio} 
        \pstart
        \ledsidenote{\textbf{5}}
        2osecundo potest argui per comparationem loci ad locabile  nam si esset aliquod mobile localiter distans infinite a suo loco in nullo tempore finito ymmoimmo numquam posset moveri in illum locum nisi per formam infinitae virtutis modo loquendo proportionaliter de loco spirituali deus est locus spiritualis animarum infinite distans ab anima propter infinitam distantiam obiectorum igitur anima non potest eum attingere cognitive nisi per notitiam infinitam simpliciter quae est nobis impossibilis simpliciter
        \pend
      
        \bigskip
         \section*{Quarta ratio} 
        \pstart
        \ledsidenote{\textbf{6}}
        deinde probatur etiam nec adiutorio luminis gloriae posset anima ad huiusmodi notitiam seu cognitionem ipsius dei elevari quia vel huiusmodi lumen esset divina essentia vel aliquod creatum sufficienter elevans potentiam animae cognitivam ad huiusmodi cognitionem dei non potest dici quod sit divina essentia quia tunc infinite immutaret animam
        \pend
     
        \pstart
        \ledsidenote{\textbf{7}}
        pro cuius determinatione supponitur quod quanto aliqua notitia est perfectioris speciei tanto perfectius immutat patet illud clare quare inductive fundatur etiam in ratione quia perfectio notitiae sumitur penes hoc quod perfectori modo in mutatimmutat Ita quod a priori perfectio noticiaenotitiae attenditur penes hoc quod est perfectioris speciei et a posteriori perfectio speciei attenditur penes perfectius in mutareimmutare
        \pend
     
        \pstart
        \ledsidenote{\textbf{8}}
        modo considerata tota latitudine cognitionum creaturarum quae in infinitum procedit ascendendo secundum perfectionem specificam notitia ipsius dei est supra totam latitudinem noticiarumnotitiarum creatarum et est perfectioris speciei quam tota latitudo noticiarumnotitiarum creaturarum et tamen tota illa latitudo infinite immutat igitur notitia dei perfectiori modo quam infinite in mutaretimmutaret
        \pend
     
        \pstart
        \ledsidenote{\textbf{9}}
        si daretur ista est fortior ratio tenentium quod deus non potest esse notitia creaturae et supplere vicem speciei et volitionis  et \name{magister iohannes de rippa}\index[persons]{} dicit quod deus supplens vicem speciei et volitionis est notitia infinita quoad speciem et infinita quoad gradum
        \pend
     
        \pstart
        \ledsidenote{\textbf{10}}
        Sed mihi videtur cum sui reverentia quod istud non sit multum bene dictum in materia quia eo ipso quod est infinita \ledsidenote{R30ra}quoad speciem est infinita simpliciter et sic infinita quo adquoad gradum sed non oporteret econverso si esset infinita quo adquoad gradum quod esset infinita simpliciter  si enim esset infinita albedo quoad gradum non propter hoc esset infinita simpliciter verbi gratia anima intellectiva est praecise finita et tamen in latitudine entium esset perfectior quam infinita albedo si daretur ideo solutio non videtur vera quia sicut tactum est perfectio notitiae simpliciter magis attenditur quo adquoad speciem quam quo adquoad gradum modo per ipsum notitia divina supplens vicem speciei est infinita quo adquoad speciem ergo infinita simpliciter  igitur infinite in mutatimmutat nec potest dici quod lumen gloriae elevans animam sit aliquod creatum ex eo quod divina essentia est immensa et incomprehensibilis a creatura igitur non potest esse aliqua species vel notitia creata ipsam essentiam divinam secundum quodlibet sui repraesentans
        \pend
     
        \pstart
        \ledsidenote{\textbf{11}}
        Et in istis rationibus quandocumque inducam de notitia beatifica quandoque de notitia hic in via et videbitur in solutione istarum rationum de utraque notitia 
        \pend
      
        \bigskip
         \section*{Quinta ratio} 
        \pstart
        \ledsidenote{\textbf{12}}
        Item 5oquinto notitia talis est simpliciter infinita Istud patet proportionando notitias secundum proportionem obiectorum nam generaliter perfectioris obiecti est perfectior cognitio et specialiter loquendo de notitia intuitiva modo latitudo notitiarum creatarum est infinita scilicet continue ascendendo secundum notitiarum perfectionem quia quacumque data potest dari in duplo perfectior in triplo et sic in infinitum sicut contingit de obiectis ipsarum notitiarum modo obiectum increatum scilicet ipse deus est perfectior quam tota latitudo obiectorum creatorum igitur notitia ipsius erit perfectior quam notitia infinitorum obiectorum creatorum et tamen notitia infinitorum obiectorum creatorum si esset esset infinita igitur etc
        \pend
     
        \pstart
        \ledsidenote{\textbf{13}}
        Et si dicatur quod non est eadem proportio cognitionum inter se sicut obiectorum ita quod non semper oportet quod obiecti perfectioris in duplo  sit notitia perfectior in duplo  retorquetur ratio iterum verbi gratia datis duabus notitiis quarum una est dupla ad aliam  si dicas quod ad causandam 2amsecundam non sufficit obiectum duplae perfectionis si enim non sufficiat capiam obiectum in centuplo vel millecuplo perfectius quod sufficiet causare notitiam in duplo perfectiorem et sic augendo continue proportiones obiectorum intantum quod semper duplabitur notitia cum sit processus in infinitum in obiectis continue ascendendo secundum perfectionem specie  et ita erit processus in infinitum in notitiis \ledsidenote{R30rb}semper duplando notitias et in multis habet locum ista augmentatio proportionum
        \pend
      
        \bigskip
         \section*{Sexta ratio} 
        \pstart
        \ledsidenote{\textbf{14}}
        Item talis dei notitia esset perfectior quam esset una alia quae esset infinitorum obiectorum creatorum et cuiuslibet distincte sed illa esset infinita igitur etc Et istam rationem facit idem quod autem notitia ipsius dei esset perfectior quam notitia infinitorum obiectorum creatorum et cuiuslibet distincte si essent patet quia deus est perfectius obiectum in se quam tota multitudo infinitorum obiectorum creatorum si esset
        \pend
      
        \bigskip
         \section*{Septima Ratio} 
        \pstart
        \ledsidenote{\textbf{15}}
        7moseptimo sic quia per talem cognitionem intuitivam deus apparet bonum finitum vel infinitum non potest dici quod apparet bonum finitum quia tunc aliqua creatura posset apparere intuitive melior deo et consequenter diligibilior et consequenter sic videns posset rationaliter Deo praeferre creaturam non potest dici quod apparet bonum infinitum quia tunc comprehenderetur quia cognosceretur secundum quodlibet sui
        \pend
      
        \bigskip
         \section*{Octava ratio} 
        \pstart
        \ledsidenote{\textbf{16}}
        Item potest argui per comparationem distantiae localis obiecti sensibilis a sensu nam propter magnam distantiam sensus decipitur circa obiectum proprium et iudicat rem minorem esse quam sit ita proportionaliter de visu intellectuali propter infinitam distantiam decipitur circa obiectum suum infinite ab eo distans et per consequens non possumus habere veram noticiamnotitiam de ipso deo
        \pend
     
        \pstart
        \ledsidenote{\textbf{17}}
        Item ex parte obiecti si esset aliquod luminosum infinitum quod infinite distaret a visu visus non iudicaret ipsum esse infinitum igitur ita erit de visu spirituali
        \pend
     
        \pstart
        \ledsidenote{\textbf{18}}
        Item etiam ex parte potenciaepotentiae  nam debilitas potentiae facit apparere obiectum remissius vel et minus quam sit ergo si potentia in infinitum distet ab obiecto in infinitum defficietdeficiet a veri obiecti cognitione illud patet de oculo male disposito cuius lux solis videtur remississima Istud etiam patet de oculo nocti coracis  unde propter debilitatem sui visus iudicat lumen solis remississimum intantum quod iudicat ipsum non sufficere ad videndum
        \pend
     
        \pstart
        \ledsidenote{\textbf{19}}
        Item dato quod deus esset ymaginabilisimaginabilis vel conceptibilis non sequitur quod sit sicut enim non sequitur  vacuum est ymaginabileimaginabile vel chimera vel inpossibileimpossibile igitur vacuum est etc quia sicut dicit \name{philosophus}\index[persons]{Aristotle} 3o \worktitle{physicorum}\index[works]{Physics} circa finem imaginatum non est credendum et da per consequens dato quod primae rationes nam concluderent quin deus esset ymaginabilisimaginabilis vel conceptibilis non propter hoc sequitur quod deus esset omnis enim ratio facta ad probandum \enquote*{quod Deus sit} est ita apparenter solubilis sicut ipsa apparenter arguit Ita quod considerata tota latitudine apparentiarum solutionum illarum rationum ex alia \ledsidenote{R30va}parte latitudo apparentiarum rationum in solutionibus probantium \enquote*{Deum esse} ex una parte et considerata tota latitudine apparentiarum solutionum illarum rationum ex alia parte  latitudo apparentiarum in solutionibus esse aequalis latitudini apparentiarum rationum probantium \enquote*{deum esse} et per consequens una pars non debet magis movere quam alia
        \pend
      
        \bigskip
         \section*{Nona ratio} 
        \pstart
        \ledsidenote{\textbf{20}}
        Item \enquote*{deum esse} traditur ad credendum igitur non potest probari quia si posset evidenter probari non exiret limites animae creatae nec etiam in vestigationisinvestigationis humanae et per consequens non esset articulus fidei quia fides est de hiis quae excedunt facultatem humanam
        \pend
      
        \bigskip
         
        \pstart
        \ledsidenote{\textbf{21}}
        Aliae sunt rationes quae tangunt alias materias quae reservabuntur loco suo et istae sufficiant pro praesenti
        \pend
       
        \bigskip
         \section*{Divisio Quaestionis} 
        \pstart
        \ledsidenote{\textbf{22}}
        Nunc autem restat aliqualiter declarare materiam pro cuius declaratione et veritatis circa 3amtertiam distinctionem quae ordine doctrinali debet esse prima licet propter divisionem libri \name{magistri}\index[persons]{Peter Lombard} primo loco non posuerit et bene secundum intentionem suam etc irca igitur materiam huius 3aetertiae distinctionis et inquisitionem veritatis erunt  pro praesenti aliqui articuli et erunt sex primus erit utrum de deo possit haberi notitia incomplexa hic in via  Secundus erit de noticiarumnotitiarum intuitione et abstractione  differentia 3us erit utrum aliquod praedicatum de deo et creaturis univoce praedicetur  et 4us erit utrum deus reponatur sub aliquo decem praedicamentorum  5us articulus erit  utrum \enquote*{deum esse} sit hic in via demonstrabile  6us erit de responsionibus ad rationes iam factas
        \pend
      
        \bigskip
         \section*{Articulus Primus} 
        \bigskip
         \section*{Circa naturalis communicatio rerum} 
        \pstart
        \ledsidenote{\textbf{23}}
        Quantum ad primum articulum resolvendo materiam ad divinam providentiam quam praemisi pro 2osecundo principio theologico omnia debite gubernantem  Sciendum est quod naturalis communicatio rerum est neccessarianecessaria pro universi conservatione maxime propter animalia ut fugiant disconvenientia et convenientia prosequantur  ymmoimmo secundum \name{magistrum}\index[persons]{Peter Lombard} 2o huius et secundum veritatem omnia sunt facta propter hominem ymmoimmo etiam angeli sunt facti propter hominem licet sint speciei superioris et ant nobilioris caelum et terra et omnis creatura elementa propter mixta mixta propter animalia et animalia propter hominem
        \pend
     
        \pstart
        \ledsidenote{\textbf{24}}
        Secundo sciendum est quod res non possunt in mediateimmediate communicari seipsis nec cognosci eo quod sensibile positum supra sensum non facit sensationem ut dicit \name{philosophus}\index[persons]{Aristotle} 2o \worktitle{de anima}\index[works]{de Anima} ubi etiam habetur quod lapis non est in anima sed species lapidis Ideo divina providentia ordinate res habent adinvicem communicari mul\ledsidenote{R30vb}tiplicando species et consequenter se ipsas quantum possunt communicando potentiis cognitivis ipsarum quia ut tactum est ut servetur debitus ordo res adinvicem communicari oportet  et quia non possunt per se ideo est instinctum eis quod se diffundant per suas species quantum possunt ideo res sunt sui ipsius diffusivae specialiter causando in potentiis sensitivis ipsarum rerum cognitiones  Et consequenter ascendendo ad intellectum et alias potentias interiores et etiam ex divina providentia potentiae provisum est de diversis organis secundum diversitatem obiectorum recipientibus species ipsorum quia unus sensus mediante uno organo non posset de omnibus obiectis iudicare sicut visus non iudicat de odoribus et saporibus consequenter ordinavit unum sensum qui habeat iudicare de obiectis aliorum sensuum et ille est sensus communis et demum ordinavit potentiam reservativam  deinde potentiam superiorem animae quae non indiget organo et ista est multiplex non quin sit una et eadem res sed habet diversos actus nam ipsa inquantum est actualis speciebus vocatur memorativa seu memoria  et ipsa est prima pars ymaginisimaginis correspondens patri in divinis qui vocatur memoria 2a  ideo inquantum fecundatur per speciem intelligibilem vocatur memoria inquantum productiva intellectionis vocatur intellectus et inquantum productiva volitionis vocatur voluntas Consequenter sciendum est quod de intellectiva est ad propositum quod ipsa est multiplex ipsa primo est potentia simpliciter apprehensiva incomplexae deinde compositiva  deinde iudicativa deinde assertiva deinde abstractiva deinde discursiva  et secundum hoc diversificantur species notitiarum et possent sic sub dividisubdividi sicut potentiae
        \pend
      
        \bigskip
         \section*{[Duae difficultates]} 
        \pstart
        \ledsidenote{\textbf{25}}
        Consequenter videndum est unde consurgit quod aliqua notitia sui obiecti repraesentativa sit  2osecundo videndum est quomodo obiectum sit sui ipsius intentionaliter diffusivum Ita quod hic sunt duae difficultates una est ex qua radice notitia dicitur sui obiecti esse repraesentativa  utrum hoc sit quia est similitudo sui obiecti vel ex natura specifica vel ex qua alia causa 2asecunda est qualiter obiectum posset intentionaliter diffundere et multiplicare sic species ipsius repraesentativas et primo dicam unum verbum de secundo et non finiam 2osecundo dicam de primo et finiam in lectione sequenti quia intendo aliqua dicere quae non reperio in scriptis etc
        \pend
      
        \bigskip
         \section*{[Circa secundam difficultatem]} 
        \bigskip
         \section*{[Opinio Ioannis de Ripa]} 
        \pstart
        \ledsidenote{\textbf{26}}
        Quantum ad 2msecundam sciendum est quod \name{magister ioannes de rippa}\index[persons]{John of Ripa} \ledsidenote{R31ra}respondet quod obiectum ymmoimmo quod nullum obiectum materiale potest concurrere obiective ad causandum huiusmodi notitiam intuitivam in potentia cuius principalis radix est quia dicit ipse nulla res agit ultra gradum proprium igitur nulla res materialis agit aliquid in materialeimmateriale cuiusmodi sunt praedictae notitiae intuitivae ut videtur et ad hoc facit tres rationes sed michimihi videtur quod omnes suae rationes fundantur in hoc quod nichilnihil agit ultra gradum proprium
        \pend
     
        \pstart
        \ledsidenote{\textbf{27}}
        unde sua prima ratio stat in hoc quod nullum obiectum materiale potest immaterialiter agere ergo non potest producere huiusmodi notitiam quia videtur esse in materialisimmaterialis
        \pend
     
        \pstart
        \ledsidenote{\textbf{28}}
        2asecunda ratio nichilnihil agit perfectius quam q sit ipsum agens modo quodlibet materiale ut dicit est perfectius materiali
        \pend
     
        \pstart
        \ledsidenote{\textbf{29}}
        3otertio dicit quod si sic tunc obiecta materialia in sacramento altaris possent agere in organis corporis christi ibidem existentis calefaciendo frigefaciendo quod non conceditur
        \pend
     
        \pstart
        \ledsidenote{\textbf{30}}
        Secunda conclusio quam ponit quod nec qualitas materialis agit intensionaliter producendo notitiam intuitivam sui ipsius quia nichilnihil agit ultra gradum proprium ipse enim incipit a superiori quia primo dicit quod non quaelibet substantia in materialisimmaterialis potest concurrere ad sui notitiam obiective et consequenter dicit quod nec qualitas materialis et ut brevius tangam radicem suam ipse supponit tria
        \pend
     
        \pstart
        \ledsidenote{\textbf{31}}
        primum est quod tota latitudo accidentium sit simpliciter infinita  Ita quod tota latitudo specierum cognotionumcognitionum producibilium est simpliciter finita in genere accidentium  2osecundo supponit quod semper obiecto alterius speciei correspondet notitia alterius speciei 3otertio supponit quod species cognitionum sunt sicut numeri
        \pend
     
        \pstart
        \ledsidenote{\textbf{32}}
        Istis dictis ymaginaturimaginatur sic et capit obiectum infinitum quod causet notitiam sui obiective in potentia  deinde capiatur obiectum perfectius causabit notitiam nobiliorem et sic ascendendo et sic dabitur certus numerus et sic data infima continue ascendendo ex quo tota latitudo est solum finita tandem devenietur ad suppremamsupremam iam datam quod non est dicendum
        \pend
     
        \pstart
        \ledsidenote{\textbf{33}}
        Item dato quod species non se haberent sicut numeri ex quo totum genus accidentium est finitum non posset infinitum procedi et sic tandem deveniretur ad suppremamsupremam notitiam vel diceretur quod proportionaliter sicut obiecta crescunt secundum perfectionem ita et notitiae ergo etc
        \pend
     
        \pstart
        \ledsidenote{\textbf{34}}
        Quantum autem est de me ego tenebo conclusionem contrariam respondendo ad istas rationes et ad primum dubium motum scilicet an notitia ex sui natura sit sui obiecti repraesentativa et sic sua ymaginatioimaginatio prima habet satis magnam apparentiam concessis suppositionibus  sed negando quod species non se haberent quod sintsicut numeri  illa ymaginatioimaginatio non procedet igitur
        \pend
        
        \endnumbering
        
     
        \end{document}
    
